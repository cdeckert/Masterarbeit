\section{SOLID Prinzipien}

Das Projekt wurde nach den SOLID-Design-Prinzipien \cite{unclebob1995} aufgebaut. SOLID steht für Single responsibility, Open-closed, Liskov substitution, Interface segregation and Dependency inversion. Es sind  eine Menge von Prinzipien, die bei der objektorientierten Programmierung zum Einsatz kommen und durch deren Anwendung die Wartbarkeit und Erweiterbarkeit des Codes erhöht werden soll. Alle fünf Prinzipien finden Anwendung in der Implementierung.

\begin{itemize}
\item Das Prinzip der Single Responsibility  ist auch unter dem Namen Kohäsion bekannt. Es sagt aus, dass eine Klasse immer nur eine Verantwortlichkeit besitzt.  Potenzielle Anforderungsveränderungen lassen sich schnell umsetzen, da potenziell nur eine Klasse von einer solchen Änderung betroffen ist.

\item Das Open/closed Prinzip sagt aus, dass Klassen für Erweiterung offen, jedoch verschlossen für Modifikation sein sollen. Dieses Prinzip erlaubt es Änderungen an einer übergeordneten Klasse durchzuführen, die automatisch von allen untergeordneten Klassen übernommen werden können. Gemeinsame Codebasen verringern den Entwicklungsaufwand und erleichtern die Erweiterung.

\item Das Liskov substitution Prinzip sagt aus, dass eine Klasse, die von einer anderen Klasse erbt, sich genau so verhalten soll wie es von der übergeordneten Klasse erwartet wird. Die Instanz eines Subtypen sollte sich von außen betrachtet also genau so verhalten, wie eine Instanz des Haupttypen.

\item Das Interface-Segregation-Prinzip dient zur Aufteilung von Interfaces. Es ist dabei besser für jeden Anwender einer Klasse ein eigenes Interface zu erstellen, das genau die Funktionen abbildet, die von einem Anwender verwendet werden, als eine großes Interface zu erstellen, das von allen Anwendern genutzt werden kann. So kann der Überblick über verschiedene Anwender besser behalten werden.

\item Das Dependency-Inversion-Prinzip sorgt für die Reduktion von Abhängigkeiten über verschiedene Komponenten eines Systems. Konkret schlägt das Prinzip vor, dass High-Level Module nicht abhängig von low-level Modulen sein sollten.

\end{itemize}

Jedes der vorgestellten Prinzipien gibt eine Leitlinie vor, mit deren Hilfe ein hohes Maß an Zuverlässigkeit, Wartbarkeit und Verständlichkeit erzielt werden kann. Nur wenn alle Prinzipien gleichermaßen zur Anwendung kommen, können diese Qualitäten gesteigert werden. Die im Weiteren beschriebene Implementierung fußt auf diesen fünf Prinzipien. In jedem Modul, jeder Klasse und jeder Methode wurde auf die Anwendung der Prinzipien geachtet.