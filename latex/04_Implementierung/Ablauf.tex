\section{Der Optimierungsvorgang}


Die Ausführung des Programms geschieht in drei Schritten (vgl. Abb. \ref{Ablauf}): (1) Konfiguration, (2) Generierung von semantisch gleichen Plänen, (3) Finden des optimalen Plans.

\begin{figure}[h]
  \centering
  \includegraphics{04_Implementierung/00_media/Ablauf.pdf}
  \caption{Ablauf der eigenen Implementierung}
  \label{Ablauf}
\end{figure}


Im ersten Schritt, wird auf Grund von externen Parametern das System konfiguriert. Die Konfiguration erfolgt durch ein JSON File. In ihm werden die Parameter (Relationen und deren Kardinalität, Join-Kanten und deren Selektivität, initaler Plan und Regelsets) für die Optimierung festgelegt. Mit Hilfe der Relationen und deren Kardinalität können später im Zusammenspiel mit Join-Kanten und Selektivitität die Kosten für einen Plan berechnet werden. Der initiale Plan dient als Startpunkt der Transformation. Auf ihn werden die Regelmengen angewendet und so logische  Äquivalente erzeugt.

Zu Beginn des zweiten Schritts, der Erzeugung von äquivalenten Plänen, wird die Zeitmessung gestartet.  Mit Hilfe von Enumeratoren werden die unterschiedlichen Regelmengen auf den initialen Plan angewendet.

In einem finalen Schritt wird findet die Kostenberechnung statt und aus den möglichen Plänen wird der günstigste ausgewählt. Diese Preisberechnung findet im Modul der Kostenschätzung statt.


