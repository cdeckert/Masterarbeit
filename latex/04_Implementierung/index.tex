\chapter{Design und Implementierung}

In diesem Kapitel wird die Implementierung des Plan Generators näher erläutert. Begonnen wird mit 5 Design-Prinzipien, die für die Entwicklung zum Einsatz gekommen sind. Sie erlauben es den Plan Generator schnell zu erweitern, Adaptoren für externe Systeme herzustellen und so die Erweiterbarkeit zu garantieren. Um einen Überblick über die unterschiedlichen Komponenten zu ermöglichen, ist in 4.2 der Ablauf des Prototypen beschrieben. Die einzelnen Software Komponenten werden daraufhin im Detail erläutert und mit konkreten Code-Beispielen verständlich gemacht.


\section{SOLID Prinzipien}

Das Projekt wurde nach den SOLID-Design-Prinzipien \cite{unclebob1995} aufgebaut. SOLID steht für Single responsibility, Open-closed, Liskov substitution, Interface segregation and Dependency inversion. Es sind  eine Menge von Prinzipien, die bei der objektorientierten Programmierung zum Einsatz kommen und durch deren Anwendung die Wartbarkeit und Erweiterbarkeit des Codes erhöht werden soll. Alle fünf Prinzipien finden Anwendung in der Implementierung.

\begin{itemize}
\item Das Prinzip der Single Responsibility  ist auch unter dem Namen Kohäsion bekannt. Es sagt aus, dass eine Klasse immer nur eine Verantwortlichkeit besitzt.  Potenzielle Anforderungsveränderungen lassen sich schnell umsetzen, da potenziell nur eine Klasse von einer solchen Änderung betroffen ist.

\item Das Open\/closed Prinzip sagt aus, dass Klassen für Erweiterung offen, jedoch verschlossen für Modifikation sein sollen. Dieses Prinzip erlaubt es Änderungen an einer übergeordneten Klasse durchzuführen, die automatisch von allen untergeordneten Klassen übernommen werden können. Ein gemeinsames Grundgerüst an Code verringern den Entwicklungsaufwand und erleichtern die Erweiterung.

\item Das Liskov Substitution Prinzip sagt aus, dass eine Klasse, die von einer anderen Klasse erbt, sich genau so verhalten soll wie es von der übergeordneten Klasse erwartet wird. Die Instanz eines Subtypen sollte sich von außen betrachtet also genau so verhalten, wie eine Instanz des Haupttypen.


\item Das Interface-Segregation-Prinzip dient zur Aufteilung von Interfaces. Es ist dabei besser für jeden Anwender einer Klasse ein eigenes Interface zu erstellen, das genau die Funktionen abbildet, die von einem Anwender verwendet werden, als ein großes Interface zu erstellen, das von allen Anwendern genutzt werden kann. So kann der Überblick über verschiedene Anwender besser behalten werden.

\item Das Dependency-Inversion-Prinzip sorgt für die Reduktion von Abhängigkeiten über verschiedene Komponenten eines Systems. Konkret schlägt das Prinzip vor, dass High-Level Module nicht abhängig von low-level Modulen sein sollten.
\end{itemize}

Jedes der vorgestellten Prinzipien gibt eine Leitlinie vor, mit deren Hilfe ein hohes Maß an Zuverlässigkeit, Wartbarkeit und Verständlichkeit erzielt werden kann. Nur wenn alle Prinzipien gleichermaßen zur Anwendung kommen, können diese Qualitäten gesteigert werden. Die im Weiteren beschriebene Implementierung fußt auf diesen fünf Prinzipien. In jedem Modul, jeder Klasse und jeder Methode wurde auf die Anwendung der Prinzipien geachtet.

\section{Überblick über den Optimierungsvorgang}
Die Ausführung des Programms geschieht in drei Schritten: (1) Konfiguration, (2) Generierung von semantisch gleichen Plänen, (3) Finden des optimalen Plans.

Im ersten Schritt, wird auf Grund von externen Parametern das System konfiguriert. Die Konfiguration erfolgt durch ein JSON File. In ihm werden die Parameter für die Optimierung festgelegt:

\begin{itemize}
\item Relationen und deren Kardinalität 
\item JoinEdges und deren Selektvität
\item Inital-Plan
\item Regelsets
\end{itemize}

Mit Hilfe der Relationen und deren Kardinalität kann später im Zusammenspiel mit JoinEdges und Selektvitität die Kosten für einen Plan berechnet werden. Der initale Plan dient als Startpunkt der Transformation. Auf ihn werden die Regelsets angewendet und so logisch äquivalente erzeugt.

Zu Beginn des zweiten Schritts, der Erzeugung von äquivalenten Plänen, wird die Zeitmessung gestartet.  Mit Hilfe von Exektuoren werden die unterschiedlichen Regelsets auf den Initialen Plan angewendet. Für jeden Knoten wird zuerst festgestellt, ob er bereits expanded wurde. Falls das nicht der Fall ist, wird geprüft, ob eine Regel aus dem Regelset anwendbar ist, ist das der Fall, kommt die Regel zum Einsatz. Nur wenn der neu-erzeugte Plan bisher noch unbekannt ist, wird er einer Äquivalenzklasse hinzugefügt.

In einem finalen Schritt wird findet die Kostenberechnung statt und aus den möglichen Plänen wird der günstigste ausgewählt. Diese Preisberechnung findet im Modul der Kostenschätzung statt.


\begin{figure}[h]
  \centering
  \includegraphics{04_Implementierung/Ablauf.pdf}
  \caption{Ablauf der eigenen Implementierung}
  \label{Ablauf}
\end{figure}


\subsection{Unterschiede im Ablauf zu Volcano und Pyro}
\section{Services}
\subsection{Zeitmessung}

\begin{figure}[ht]
  \centering
  \includegraphics{04_Implementierung/00_media/Stopwatch.pdf}
  \caption{Klassendiagramm: Stopwatch}
  \label{ClassStopwatch}
\end{figure}

Wie bereits in Abb. \ref{Ablauf} dargestellt, wird die Zeitmessung, die durch die Klasse Stopwatch (vgl. Abb. \ref{ClassStopwatch}) durchgeführt wird, nach der Konfiguration zuerst instantiiert und dann mit der Methode \texttt{start()} gestartet. Sobald alle Pläne expandiert und die Kosten berechnet sind, wird die Messung durch die Methode \texttt{stop()} beendet. Um eine möglichst genaue Messung durchführen zu können, wurde die Uhr \texttt{std::chrono::steady\_clock} verwendet. Diese Uhr ist Teil von C++11. Wie in der Dokumentation \cite{cppreference_2015_clock} beschrieben, handelt es sich bei der Uhr um eine monotone Uhr. Sie kann nicht rückwärts laufen, solange die physische Zeit fortfährt. Die Uhr selbst mit der Wall\-Clock\-Time verbunden, ist das passende Werkzeug zur Messung von Intervallen. 

Die Dauer zwischen Start der Expansion und dem Ende der Kostenberechnung wird in Nanosekunden gemessen und mit der Methode \texttt{getDuration()} ausgegeben, um ein möglichst akkurates Ergebnis zu erhalten. Die gemessenen Ergebnisse werden sowohl als Debug-Log in der Konsole ausgegeben, als auch  in einem Log File gespeichert.




\subsection{Logging und Debugging}
Zum Debugging wurde auch eine externe Bibliothek eingesetzt: EasyLogging++ \cite{easylogging}, ein einfach zu bedienendes, jedoch hochgradig konfigurierbares Logging Instrument. Gerade die Leichtgewichtigkeit - das Tool besteht nur aus einer Headerklasse -, die einfache Bedienung und die Geschwindigkeit gaben  den Ausschlag zur Nutzung der Library. Mit Hilfe dieser Library werden debugging Informationen geschrieben, aber auch Zeitmessungen gespeichert, Pläne ausgegeben und sonstige Debugging Nachrichten ausgegeben. Insbesondere ist die Unterscheidung zwischen unterschiedlichen Debug\-Leveln wichtig. Auf mehreren Ebenen (INFO, WARNING, DEBUG, etc.) können Informationen ausgegeben werden. Je nachdem welches Level angesprochen ist, werden nur Informationen über dieses ausgegeben. 

\subsection{Operatoren und Operations}
Wie in Abb. \ref{PlanNodeClass} zu sehen, sind zwei Operatoren vorgesehen JOIN und SCAN. Um aus ihnen im Zusammenhang mit Planknoten und Äquivalenzklassen Pläne herzustellen, wurde die Klasse \texttt{Operations} (vgl. \ref{ClassOperations} erstellt. Diese Klasse bietet einfachen Zugriff auf die meist verwendeten Arten von Äquivalenzklassen und Planknoten. Wird beispielsweise mit der Methode \texttt{scan} direkt eine Äquivalenzklasse mit einer bestimmten Relation, ein übergeordneter Planknoten und eine wiederum übergeordnete Äquivalenzklasse erstellt werden.

Neben diesen Operator steht auch ein Oprator zur Erzeugung von Joins mit und ohne übergeordneter Äquivalenzklasse zur Verfügung \texttt{join} resp. \texttt{joinPN}.

Um Kosten bei der Erzeugung neuer Objekte zu sparen, werden alle Planknoten und Äquivalenzklassen in Resourvars angelegt. Es werden also bereits vor Start des eigentlichen Tests alle Instanzen gebildet und während der Laufzeit mit konkreten Daten gefüllt.



%\input{04_Implementierung/Organisation.tex}
\section{Konfiguration}


\begin{figure}[ht]
  \centering
  \includegraphics[width=1\textwidth]{04_Implementierung/00_media/Tool.png}
  \caption{Konfigurationstool: Scenario-Generator}
  \label{ScenarioGenerator}
\end{figure}

Das Konfigurationsmodul besteht aus zwei Komponenten. Auf der einen Seite eine Javascript-HTML Anwendung zur Generierung von Test-Szenarien (vgl. Abb. \ref{ScenarioGenerator}). Mit Hilfe dieser Anwendung können Konfigurationsfiles für den späteren Test erstellt werden. Dieses Konfigurationssystem, das in  zu sehen ist, bietet dem Nutzer die Möglichkeit Relationen mit Kardinalitäten zu erzeugen, Selektivitäten festzulegen und schlussendlich einen initialen Plan zu erzeugen. Ebenfalls ist es möglich die verschiedenen Regelmengen, die geprüft werden sollen, festzulegen.

Das Ergebnis dieses Moduls ist eine Konfigurationsdatei im JSON-Format. Das Tool unterstützt die Erzeugung mehrerer Szenarien in einem Konfigurationsfile. So können mehrere Szenarien in einem Test-Run durchgespielt werden. Die Verwendung des JSON-Formats bietet viele Vorteile. U.a. waren die folgenden Vorteile ausschlaggebend für die Verwendung von JSON:

\begin{itemize}
\item Im Gegensatz zu XML ist JSON einfacher. Weniger Grammatik ist vorhanden.
\item Das Format ist hoch interoperable und kann nativ von Javasciript ausgegeben werden. Ebenfalls ist es möglich mit Hilfe von einfachen Parsern JSON in C++ zu verarbeiten.
\item JSON ist durch die einfache, jedoch standardisierte Form leicht für den Menschen zu lesen und Fehler sind leichter aufzuspüren.
\item Das gewählte Format zur Repräsentation der unterschiedlichen Parameter ist einfach erweiterbar. Attribute und neue Objekte können jederzeit hinzugefügt werden.
\end{itemize}

\subsection{Aufbau und Funktion der Konfigurationsdatei}

Ein konkretes Test File ist in \ref{JsonConfigFile} zu sehen. Jedes Test File ist ein Array von Test-Objekten. Für jeden Test wird mit dem Attribut \texttt{name} ein Name gespeichert. Für jede Relation, die in einem Test vorkommt, wird ein Relations-Objekt erzeugt. Es besteht aus einer \texttt{id}, der Relations-Id, und der dazugehörigen Kardinalität. Mehrere dieser Relations-Objekte werden in einem Array gespeichert und sind dem \texttt{relation}-Attribut des Test-Objekts zugewiesen. Im konkreten Fall sind zwei Relationen vorhanden. Eine mit der \texttt{id:1} und eine mit der \texttt{id:2} 2. Für Beide wurde eine Kardinalität zugewiesen. 

Auch für Join-Kanten werden Objekte erzeugt. Ein Join-Kanten Objekt bezeichnet eine Kante zwischen zwei Relationen. Eine Kante könnte, wie im Beispiel zu sehen von der Relation 1 zu der Relation 2 verlaufen, wobei die Richtung nicht entscheidend ist. Einer Join-Kante wird auch eine Selektivität zugeordnet. Mehrere dieser Join-Edge-Objekte werden als Array im Attribut \texttt{joinEdges} hinterlegt.

Neben Join-Kanten und Relationen sind auch die zu testendeden Algorithmen gespeichert. Sie werden in einem Array abgelegt und sind dem Attribut \texttt{algorithms} zugewiesen. Im vorliegenden Beispiel werden die Regelsets \texttt{RS-B0}, \texttt{RS-B1} und \texttt{RS-B2} getestet.

Auch ein initialer Plan wird im Test-Objekt gespeichert. Ein Baum besteht immer aus Knoten. Für jeden Knoten wird ein Operator im Attribut \texttt{op} gespeichert. Es wird der Operator \texttt{SCAN} und der Operator \texttt{JOIN} unterstützt. Ein Knoten kann eine linke und eine rechte Seite haben. In Zeile 14 ist eine Scan Operation zu sehen. Es wird nur die linke Seite verwendet. Im Attribut \texttt{l} ist die ID der Relation abgelegt, die zu scannen ist. Bei \texttt{JOINs} wird auch die rechte Seite verwendet (vgl. Zeile 16). In den Attributen \texttt{l} und \texttt{r} können entweder weitere Knoten oder Relations-Ids gespeichert werden. Neben Operator, links und rechts ist das Attribut \texttt{relations} Teil des Knotens. Hier ist eine einfache Repräsentation des Knotens gespeichert. Im Falle von Zeite 14 \texttt{1}. sind komplexere Knoten und Subknoten vorhanden, können komplexere Daten im Feld \texttt{Relations} abgelegt sein. Beispielsweise ist eine solche komplexere Form in Zeile 19 ( \texttt{JOIN(1,2)}) zu erkennen. Hier ist ein Join zwischen 1 und 2 zu sehen.

\begin{minipage}{\linewidth}
\lstinputlisting[caption=JSON: Konfigurations-File, label=JsonConfigFile]{04_Implementierung/00_media/config.json}
\end{minipage}

\subsection{Konfiguration eines Test innerhalb des Prototypen}

Die eigentliche Konfiguration findet in C++ statt. Mit der Bibliothek json11 \cite{json11} wird das Konfigurationsfile gelesen. Dieser Vorgang wird von der Klasse \texttt{Config\-urator} übernommen. Auf Basis eines Dateipfades zu einer Konfigurationsdatei erzeugt die Klasse einen Vektor von \texttt{Configuration} Instanzen. Für jedes JSON Test-Objekt wird eine solche Konfigurationsinstanz erzeugt und die für die Konfiguration notwendigen Informationen hinterlegt.

Jedes Konfigurationsobjekt (vgl. Abb. \ref{Konfiguration}) bietet eine Menge unterschiedlicher Methoden an, um Informationen über den aktuellen Test zu erlangen.

Mit Hilfe der Methode \texttt{getInitaleTree()} wird ein Baum aus Äquivalenzklassen und Planknoten erzeugt, die den initalen Plan erzeugen. Die Methode erzeugt immer eine neue Instanz des gesamten Baums.

Die Kardinalität und Selektivität, die als ausschlaggebende Kennzahlen zur Berechnung des optimalen Plans herangezogen werden, sind mit den Methoden \texttt{get\-Cardinaility()} und \texttt{get\-Selectivity()} zu erfragen. Auch die getesteten Algorithmen können mit der Methode \texttt{getAlgorithms()} ausgelesen werden.


\begin{figure}[ht]
  \centering
  \includegraphics[width=0.75\textwidth]{04_Implementierung/00_media/ConfigurationClass.pdf}
  \caption{Klassendiagramm: Konfiguration}
  \label{Konfiguration}
\end{figure}

\section{Planknoten und Äquivalenzklassen}

Die unterschiedlichen Pläne werden in diesem Protypen als Äquivalenzklassen und Planknoten implementiert. Die Relationen werden mit Hilfe von Bitvektoren dargestellt. Im Folgenden wird auf die genaue Implementierung von Bitvektoren, Äquivalenzklassen und Planknoten eingegangen.

\subsection{Repräsentation von Relationen}
\label{sec:Bitvector}

Die einzelnen Relationen werden mit Hilfe eines Bitvektors dargestellt. Jeder Basis-Relation - sie repräsentiert eine physische Relation - ist eine ID zugeordnet. Eine Basis-Relation wird in diesem Prototypen mit Hilfe eines auf \texttt{TRUE} gesetzten Bits in einem Bitvektor repräsentiert. Auch mehrere Relationen können durch einen Bitvektor abgebildet werden. Um dies zu tun, werden mehrere Bits  im Bitvektor auf \texttt{TRUE} gesetzt.

\subsubsection{Implementierung}
\begin{figure}[ht]
  \centering
  \includegraphics{04_Implementierung/00_media/Bitvector.pdf}
  \caption{Bitvekotren als Repräsentation von Relationen oder Joins}
  \label{Bitvektor}
\end{figure}

Als Basis für die Implementierung dient ein Bitvektor. Ein Bitvektor sind mehrere Bits, die entweder \texttt{TRUE} also \texttt{1} oder \texttt{FALSE} also \texttt{0} sein können. Ist das n-te Bit eines Bitvektors gesetzt, so repräsentiert dieses Bit die n-te Relation. Beispielsweise bezeichnet der Bitvektor \texttt{010000000} die Relation mit der ID \texttt{1}. Mit Hilfe des Bitvektors können auch mehrere Relationen gespeichert werden. \texttt{01010000000} bezeichnet folglich die Relation mit der ID \texttt{1} und die Relation mit der ID \texttt{3}. Für die durchgeführten Berechnungen ist es nicht notwendig, dass eine Relation mit ihrem tatsächlichen Namen bekannt ist. Es reicht eine Bezeichnung mit Hilfe von Nummern aus.

Vorteil für die Verwendung von Bitvektoren ist, dass einfach Mengenoperationen durchgeführt werden können. So kann einfach geprüft werden, ob Äquivalenzklassen gemeinsame \texttt{JOIN}-Kanten besitzen oder neue Relationen einer Äquivalenzklasse hinzugefügt werden. (vgl. Abb. \ref{Bitvector}) Dies ist besonders effizient, da nur Bits und keine komplexen Objekte wie Strings verarbeitet werden müssen.



Bitvektoren kommen immer in Äquivalenzklassen zum Einsatz. Wenn es sich um eine Äquivalenzklasse handelt, die eine Basis-Relation abbildet, dann ist nur ein Bit im Vektor auf \texttt{TRUE} gesetzt. Wenn es sich hingegen um eine Äquivalenzklasse handelt, die mehrere Relationen repräsentiert, werden auch mehere Bits im Bitvektor gesetzt.



\subsubsection{Erweiterbarkeit}

Da der Bitvektor, wie viele andere Klassen auch, einen Template Parameter besitzt, kann die Datenstruktur, auf deren Basis der Vektor implementiert ist ausgetauscht werden. Beispielsweise kann anstatt eines \texttt{unsigned int} auch ein \texttt{unsigned long long} verwendet werden.

Auf der anderen Seite dient der Bitvektor als Parameter für die Planknoten-Klasse. Der gesamte Bitvektor kann also ersetzt werden ohne dabei den Code anderer Klassen anpassen zu müssen.


\subsection{Planknoten und Äquivalenzklassen}


\begin{figure}[ht]
  \centering
  \includegraphics{04_Implementierung/00_media/JoinScan.pdf}
  \caption{Planknoten und Äquivalenzklassen}
  \label{PlanAequi}
\end{figure}

\texttt{Plan\-Node} und \texttt{Equi\-valence\-Classes} sind die Datenstrukturen in der Pläne gespeichert sind. Einem Planknoten ist ein Operator zugewiesen. Beispielsweise \texttt{JOIN} oder \texttt{SCAN}. In \texttt{Equi\-valence\-Class} werden mehrere Pläne gespeichert, die alle semantisch gleich sind. Ein einfaches Beispiel ist in Abbildung \ref{PlanAequi} zu finden. Ein einfacher Plan bestehend aus einem \texttt{JOIN}- und zwei \texttt{SCAN}-Operatoren ist zu sehen. Der oberste Knoten \texttt{E3} ist eine Äquivalenzklasse. In ihr findet sich der erste Planknoten ein \texttt{JOIN}. Der \texttt{JOIN} hat zwei Seiten, eine linke und eine rechte. Beide Seiten sind mit einer Äquivalenzklasse verbunden \texttt{E1} resp. \texttt{E2}, die jeweils einen \texttt{SCAN} beinhalten, der die Basis-Relationen \texttt{R1} bzw. \texttt{R2} einliest. Die beiden Basis-Relationen sind ebenso wie die Äquivalenzklassen als \texttt{Equi\-valence\-Class} im System abgelegt.




\subsubsection{Implementierung von Äquivalenzklassen}

\begin{figure}[ht]
  \centering
  \includegraphics[width=1\textwidth]{04_Implementierung/00_media/ClassEquivalenceClass.pdf}
  \caption{Klassendiagramm: Äquivalenzklasse}
  \label{ClassEquivalenceClass}
\end{figure}


Wie bereits beschrieben, werden semantisch gleiche Planknoten in Äquivalenzklassen gespeichert und Basis-Relationen durch Äquivalenzklassen repräsentiert. Die konkrete Implementierung der Äquivalenzklasse ist in Abb. \ref{ClassEquivalenceClass} zu erkennen.

Bevor ein Plan einer Äquivalenzklasse zugeordnet werden kann oder eine Relation repräsentiert werden kann, muss die Äquivalenzklasse instantiiert werden. Bei der Instantiierung befinden sich noch keine Informationen über Planknoten oder Relationen in der Äquivalenzklasse. Die Variablen \texttt{\_first}, \texttt{\_last} und \texttt{\_best}, die den ersten, den letzten und den besten Plan anzeigen, sind auf \texttt{NULL} gesetzt. Die Bitvektoren \texttt{\_relations} und \texttt{\_neighbors} sind leer und repräsentieren noch keine Relationen. Auch die boolean variable \texttt{\_explored}, die anzeigt, ob eine Äquivalenzklasse schon vollständig erforscht ist, ist auf falsch gesetzt.


Von diesem Startpunkt aus, kann eine Äquivalenzklasse zwei Wege einschlagen: entweder eine Menge von Plänen speichern oder Basis-Relationen speichern. 

Wird eine Basis-Relation gespeichert, so kann mit der Methode \texttt{setRelations( Bitvector\_t \& )} eine Bitevektor übergeben werden. Die Nachbarschaft einer \texttt{Equi\-valence\-Class} kann mit der Methode \texttt{set\-Neighbors( Bitvector\_t \& )} festgelegt werden. Als Nachbarschaft werden die Knoten bezeichnet, mit denen eine Join-Kante besteht.




Falls die Äquivalenzklasse mehrere Pläne speichert, kann auf einen Knoten mit Hilfe der Methode \texttt{push\_back( PlanNode\_t \& )} ein Planknoten angehängt werden. Zuerst wird geprüft, ob bereits ein Planknoten vorhanden ist. Falls das nicht der Fall ist, wird der Knoten als erster, \texttt{\_first}, und als letzter, \texttt{\_last}, Knoten in der Äquivalenzklasse festgelegt. Im selben Schritt wird von dem Planknoten die Nachbar- und die vorkommenden Relationen in den dafür vorgesehenen Bitevektoren gespeichert. Falls ein Knoten schon vorhanden ist, wird auf dem letzten Knoten die Methode \texttt{setNext( PlanNode\_t \& )} aufgerufen. Diese hängt an einen Planknoten einen weiteren an. Die konkrete Implementierung hängt vom jeweils verwendeteten Planknoten ab. Die exakte Implementierung ist im Code-Beispiel \ref{listing:Push-Back} zu sehen.

\lstinputlisting[caption=C++: EquivalenceClass push\_back, label=listing:Push-Back]{04_Implementierung/00_media/Push_back.h}

Eine weitere Möglichkeit Pläne an eine Äquivalenzklasse anzuhängen ist die Methode \texttt{concat(EquivalenceClass *)} Eine andere Äuqivalenzklasse kann übergeben werden und wird automatisch an die bestehende Klasse angehängt. So können zwei Äquivalenzklassen kombiniert werden.



Ein weiterer wichtiger Teil, der gesondert hervorgehoben werden muss, ist die Methode \texttt{isOverlapping( Bitevector\_t \& )} Die Methode wird  verwendet, um zu prüfen, ob die Nachbarn einer Äquivalenzklasse in einem gegebenen Bitvektor vorkommen. Repräsentiert eine Äquivalenzklasse beispielsweise die Relation 1,2,3 und hat die Nachbarn 4, 5 und 6., so gibt die Methode \texttt{isOverlapping true} zurück, falls der Eingabe-Bitvektor 4, 5 und/oder 6 enthält. Dieser Zustand wird im weiteren auch als überlappend bezeichnet und wird genutzt um zu prüfen, ob eine Regel anwendbar ist.

Ebenfalls ist es notwendig, über die in einer Äquivalenzklasse gespeicherten Pläne zu iterieren. Eine Äuqivalenzklasse kann aussehen,  wie in Abb. \ref{EquivalenceClassList} dargestellt. Die Äquivalenzklasse zeigt mit dem Zeiger \texttt{\_first} auf den ersten Planknoten. Dieser Zeit auf den nächsten Planknoten. Auf den letzten Planknoten wird von der Äuqivalenzklasse mit dem Zeiger \texttt{\_last} gezeigt. Der beste Plan wird mit dem Zeiger \texttt{\_best} markiert. Bei der Iteration über die Pläne einer Äquivalenzklasse kommt die Methode \texttt{begin()} zum Einsatz, die einen Iterator für den ersten Plan zurückliefert. Wird die Methode \texttt{node()} auf den Iterator angewendet, wird der jeweilige Planknoten zurückgeliefert. Neben der Methode für den ersten Knoten, lässt sich auch mit \texttt{last()} das letzte Element ausgeben. Durch diese Methoden kann über die Äquivalenzklasse iteriert werden.

\begin{figure}[ht]
  \centering
  \includegraphics{04_Implementierung/00_media/EquivalenceClassList.pdf}
  \caption{Schematische Darstellung einer Äquivalenzklasse mit mehreren Planknoten}
  \label{EquivalenceClassList}
\end{figure}


\subsubsection{Implementierung von Planknoten}

\begin{figure}[ht]
  \centering
  \includegraphics[width=\textwidth]{04_Implementierung/00_media/PlanNodeClass.pdf}
  \caption{Klassendiagramm: PlanNode}
  \label{PlanNodeClass}
\end{figure}

Wie bereits beschrieben, können Äuqivalenzklassen zur Speicherung von Plänen genutzt werden. Ein Planknoten repräsentiert eine Operation und kann bis zu zwei Äquivalenzklassen beinhalten. Wie in Abbildung \ref{PlanNodeClass} zu sehen, beinhaltet der Plan-knoten Zeiger auf einen linken und einen rechten Äquivalenzknoten, sowie einen Zeiger auf den nächsten Planknoten. Ebenso ist ein Operator Teil des Planknotens. Auch sind vier boolean Variablen vorhanden: \texttt{\_commutativity\-Enabaled}, \texttt{\_left\-Associativity\-Enabled},  \texttt{\_right\-Associativity\-Enabled},  \texttt{\_exchange\-Enabled}. Die boo\-lean-Variablen werden nur für die Regelmenge \texttt{RS-B2} verwendet und werden durch den Konstruktor auf \texttt{TRUE} gesetzt. Im selben Schritt werden die Zeiger mit \texttt{NULL} initialisiert. Den Zeigern und dem Operator können später durch die Methode \texttt{set} mit konkreten Zeigern und einem Operator zugewiesen werden.

Planknoten können aneinander gehängt werden, wie in Abb. \ref{EquivalenceClassList} zu sehen. Dies geschieht mit der Methode \texttt{concat}. Falls der aktuelle Knoten nicht der letzte ist, kann mit der Methode contact (PlanNode *) zuerst zum letzten Knoten gesprungen werden und dort der neue Knoten eingefügt werden. 

Die Methoden \texttt{disable\-All\-Rules()} und \texttt{disable\-All\-And\-Enable\-Commutativity()} kommen bei der Verwendung der Regelmenge \texttt{RS-B2} zum Einsatz. Nach Anwendung einer Regel wird so im Planknoten festgelegt, dass bestimmte Regeln nicht mehr auf einen Knoten angewendet werden dürfen. Um sicherzustellen, dass die Regeln nur \texttt{is\-Left\-Associativity\-Enabled()}, \texttt{is\-Right\-Associativity\-Enabled()} und \texttt{is\-Exchange\-Enabled()} geprüft, ob eine Regel zur Anwendung kommen darf.

Um die Äquivalenzklasse eines Planknotens zu erreichen, ist es möglich mit der Methode \texttt{left()} bzw. \texttt{l()} einen Zeiger auf die entsprechende Klasse zu erhalten. Analog dazu können die Methoden \texttt{right()} bzw. \texttt{r()} verwendet werden. Mit den Methoden \texttt{has\-Left()} bzw. \texttt{has\-Right()} wird geprüft, ob ein Planknoten eine rechte bzw. linke Äquivalenzklasse besitzt. Diese Methoden sind insbesondere für die Implementierung von Regelnbedingungen wichtig.


\subsubsection{Erweiterbarkeit}

Auch Planknoten und Äquivalenzklassen sind leicht erweiterbar. Auf der einen Seite ist es möglich die \texttt{Plan\-Node} vollkommen auszutauschen. Dies ist besonders einfach möglich, da die Äquivalenzklasse \texttt{PlanNode\_t} als Template-Parameter vorsieht und somit der Weg geebnet ist einen weiteren Knoten anzufügen. Neben dieser Erweiterung ist es ebenfalls denkbar, dass mehr als zwei Äquivalenzklassen einem Planknoten zugeordnet sind. 
\section{Regeln und Regelmengen}
Auf die Planknoten müssen während der Enumeration Regeln angewendet werden. Diese Regeln sind im vorliegenden Prototypen als \texttt{Rules} und \texttt{RuleSets} abgelegt.

\subsection{Regelmengen}
Mehrere Regeln werden in einer Regelmenge zusammengefasst.
Insgesamt wurden vier unterschiedliche Regelmenegen implementiert: \textit{RS-B0}, \textit{RS-B1}, \textit{RS-B2} und \textit{GraphRule}.
Alle Mengen basieren auf den von Pellenkoft et al. vorgestellten Regelsets und dem von \ref{shanbhag2014optimizing} implementierten GraphRule.

\begin{figure}[ht]
  \centering
  \includegraphics[width=\textwidth]{04_Implementierung/00_media/RuleSets.pdf}
  \caption{Klassendiagramm: Regelmengen und Regeln}
  \label{RuleSetClass}
\end{figure}

\subsubsection{Implementierung}
\label{sec:RuleImplementation}

Alle Regelmengen erben von der Klasse \texttt{RuleSet}, die die Methode \texttt{getRules()} implementiert, mit deren Hilfe ein Vektor von Regeln ausgegeben wird. Je nach Regelmenge können andere Regeln vorhanden sein. Eine Übersicht über Regeln und deren Zuordnung zu Regelsets findet sich in Abb. \ref{RuleSetClass}.

Die einzelnen Regeln bei der Erstellung einer Regelmenge instantiiert und dem Vector \texttt{\_rules} zugeordnet. Die Sammlung der Regeln in einem Vector ist möglich, da alle Regeln von der abstrakten Klasse \texttt{Rule} erben. 

Konkret ist dem Regelset \textit{RS-B0} die Regel \texttt{RightAssociativity}, \texttt{Commutativity} und \texttt{LeftAssociativity} zugeorndet. Dem Regelset \textit{RS-B1} \textit{Left Associativity} und \textit{Commutativity}. Dem Regelset \textit{RS-B2} sind die Varianten von Kommutativität, linker und rechter Assozativität, die speziell für diese Regelmenge erstellt wurden, zugeordnet ebenso wie die \texttt{Exchange} Regel. Der \textit{Graph Rule} wird nur für das \textit{GraphRule Set} benötigt.

\subsubsection{Erweiterbarkeit}
Die Erweiterung der Regelmengen ist in verschiedenen Dimensionen möglich. Neue Funktionen können für die Regelmengen implementiert werden, ebenso ist die Erstellung neuer Regelmengen möglich.

Auf funktionaler Ebene ist es vorstellbar, dass die Reihenfolge der Regeln dynamisiert wird. Aktuell ist es nur möglich, die Regeln in Form eines Vektors auszugeben, dessen Reihenfolge immer gleich ist. Da alle konkreten Regelmengen von der selben Klasse \texttt{RuleSet} erben, ist die Implementierung einer anpassbaren Reihenfolge leicht möglich und muss für alle Klassen nur einmal vorgenommen werden.


Auch das Hinzufügen von neuen Regeln zu bestehenden Regelmengen oder die Erweiterung von bestehenden Regelmengen ist möglich. Beispielsweise kann von bestehenden Regelmengen geerbt wird bzw. neue Regeln und deren Regelmengen durch die Implementierung der standardisierten Interfaces \textit{RuleSet} entstehen.

Die Erweiterbarkeit konnte mit der Implementierung der Regelmenge  \textit{Graph Rule} unter Beweis gestellt werden, da diese Regelmenge erst später entwickelt wurde und auf die bestehende Infrastruktur aufsetzte.




\subsection{Regeln}

\begin{figure}[ht]
  \centering
  \includegraphics[height=\textwidth]{04_Implementierung/00_media/Rules.pdf}
  \caption{Klassendiagramm: Regeln}
  \label{RuleClassDiagram}
\end{figure}

Die einzelnen Regeln, die Teil der Regelmengen sind, werden jeweils in einer eignen Klasse abgelegt. Aktuell sind acht Regeln vorhanden: (1) \texttt{Commutativity}, (2) \texttt{Left Associativity}, (3) \texttt{Right\-Associativity}, (4) \texttt{Commu\-tativity B2}, (5) \texttt{Left\-Associtativity-B2}, (6) \texttt{Right\-Associativity-B2}, (7) \texttt{Exchange}, (8) \texttt{Graph Rule}. Die Zuordnung der Regeln zu den unterschiedlichen Regelmenegen wurde in \ref{sec:RuleImplementation} erklärt und in Abb. \ref{RuleSetClass} veranschaulicht.

\subsubsection{Implemenetierung}


\begin{figure}[ht]
  \centering
  \includegraphics{04_Implementierung/00_media/Plan.pdf}
  \caption{Plandiagramm: Einfacher Beispiel-Plan}
  \label{SimplePlan}
\end{figure}


Auch bei den Regeln erben alle Regeln direkt oder - und das ist neu - indirekt von der Klasse \texttt{Rule}. Die Organisation der Klassen ist in Abb. \ref{RuleClassDiagram} dargestellt. Alle Regeln implementieren das Interface, das durch die abstrakte Klasse \texttt{Rule} vorgegeben ist. Durch die Methode \texttt{getName()} wird der Name der jeweiligen Regel zurückgeliefert. Wie bei anderen Implementierungen sind die Regeln in zwei Teilen organisiert. Mit Hilfe der Methode \texttt{isApplicable} kann festgestellt werden, ob eine Regel anwendbar ist, Die Methode \texttt{apply} wendet die Regel an und liefert einen Planknoten zurück. Der Planknoten selbst, sowie die Operationen die zur Erzeugung eines neuen Plans führen, können durch Template Parameter ausgetauscht werden. So ist es möglich andere Planknoten zu verwenden und die Funktionsweise von Operationen wie \texttt{JOIN} völlig neu zu bestimmen.


Die konkrete Implementierung der Regeln lässt sich am besten am Beispiel eines konkreten Plans erläutern. Ein solch konkreter Plan ist in Abb. \ref{SimplePlan} zu sehen. Der oberste Knoten wird als \textit{parent} bezeichnet. Ihm sind zwei Äquivalenzklassen untergeordnet \textit{parent.left} und \textit{parent.right}. 

Wie bereits bekannt, kann eine Äquivalenzklasse meherere Planknoten beheimaten. Bei der Ausführung der Regeln wird ein Planknoten nach dem anderen betrachtet und für diesen eine Regel ausgeführt. In diesem Falle ist der Äquivaelnzklasse \textit{parent.left} und \textit{parent.right} je nur ein Planknoten zugeordnet. Diesen Planknoten ist selbst wieder je zwei Äquivalenzklassen untergeordnet.


Bei der Ausführung einer Regeln wird in die konkrete Regel der \textit{parent} und je ein konkreter Planknoten der untergeordneten Äuqivalenzklassen übergeben. Im konkreten Fall sind das \textit{left} und \textit{right}





\subsubsection{Implementierung von Kommutativität}

Bei der Regel Kommutativität wird nur auf den \textit{Parent} und dessen Äquivalenzklassen geachtet. Im \texttt{isApplicable}-Teil der Regel wird zuerst geprüft, ob die Operation des \textit{parent}-Knoten ein JOIN ist. Wenn das der Fall ist, wird noch geprüft, ob die linke und rechte Äquivalenzklasse überlappen. Dies geschieht mit der Methode \texttt{isOverlapping}. Liefert auch diese Methode \texttt{true} zurück, kann die Regel angewendet werden.

Bei der Regelanwendung kommt wie bei anderen Regeln die Hilfs Klasse \texttt{Operations} vor. Mit ihrer Hilfe wird ein neuer Planknoten erzeugt, der die linke und rechte Äquivalenzklasse des Planknoten vertauscht.

Die genaue Implementierung der Regel ist in \ref{CommutativityCode} zu sehen.


\lstinputlisting[caption=C++: Kommutativität, label=CommutativityCode]{04_Implementierung/00_media/Commutativity.h}


\subsubsection{Implementierung von Assoziativität}

Bei der linker Assoziativität wird zuerst geprüft, ob der Operator des \textit{parent}-Knotens ein JOIN ist. Wenn dies der Fall und der \textit{left}-Knoten ein JOIN Operator ist, dann wird geprüft, ob die rechte Äquivalenzklasse des \textit{parent}-Knoten mit der rechten Äquivalenzklasse des \textit{left}-Knotens überlappt. Falls alles zutrifft ist die Regel anwendbar.

Bei der Anwendung wird zuerst ein neuer Planknoten erzeugt, der direkt einer neuen Äquivalenzklasse zugeordnet wird. Der neue Planknoten repräsentiert den Join zwischen der rechten Äquivalenzklasse des \textit{left}-Knotens und der rechten Äquivalenzklasse des \textit{parent}-Knotens. Die neugebildete Klasse wird zum rechten Teil eines wiederum neuen Äquivalenzknotens, der den linken \textit{parent}-Teil mit der neuen Äquivalenzklasse joint. Das Ergebnis ist linke Kommutatvitität.

Diese Implementierung ist auch in \ref{LeftAssociativityCode} nachzuvollziehen und funktioniert analog zur Implementierung der rechten Assiziaitvität, die in \ref{RightAssociativityCode} zu sehen ist.

\lstinputlisting[caption=C++: Linke Assoziativität, label=LeftAssociativityCode]{04_Implementierung/00_media/LeftAssociativity.h}

\lstinputlisting[caption=C++: Right Assotiativität, label=RightAssociativityCode]{04_Implementierung/00_media/RightAssociativity.h}


\subsubsection{Implementierung von abgeleiteten RS-B2 Regeln}

Die Regeln von RS-B2 unterscheiden sich massgebnich dadurch, dass nach Anwendung einer Regeln andere Regeln von der Anwendung auf den neuen Knoten ausgeschlossen sind. Wie bereits beschrieben, ist die Information, ob eine Regel auf einen Planknoten bereits angewendet wurde im Planknoten selbst gespeichert. Die die dafür vorgesehenen boolean Variablen, können mit Hilfe der Methoden \texttt{disable\-All\-Rules()} und \texttt{disable\-All\-And\-Enable\-Commutativity()} auf \texttt{false} gesetzt werden. Mit den Methoden \texttt{is[RULENAME]Enabled()} lässt sich für jede Regel prüfen, ob diese auch benutzt werden darf.

Die konkrete Implementierung sieht vor, dass die Regeln \texttt{Commutativity\-B2, Left\-Associativity\-B2} und \texttt{Right\-Associativity} direkt von \texttt{Commutativity}, \texttt{}{Left\-Associativity} und \texttt{Right\-Associativity} erben. Bei dem Aufruf von \texttt{is\-Applicable} wird zuerst geprüft, ob die Regel \texttt{enabled} ist mit Hilfe der dafür vorgesehenen Zugriffsmethode. Wenn das der Fall ist, kann die geerbte Methode aufgerufen werden und so geprüft werden, ob auch alle anderen Voraussetzungen erfüllt sind.

Bei der Ausführung der \texttt{apply}-Methode wird die zuerst die geerbte Methode ausgeführt und dann auf diese Methode die Notwendige disable Methode aufgerufen, um für diesen Knoten die Regel zu deaktivieren.

Konkret kann diese Implementierung in \ref{CommutativityB2Code} nachvollzogen werden.

\lstinputlisting[caption=C++: Kommutativität B2, label=CommutativityB2Code]{04_Implementierung/00_media/Commutativity_B2.h}


\subsubsection{Implementierung der Exchange Rule}
Die Exchange Regel hingegen erbt nicht von anderen Regeln, sondern ist neu implementiert worden. Auch diese Implementierung sieht wieder vor, dass im \texttt{is\-Applicable}-Teil geprüft wird, ob sowohl der \textit{parent}, \textit{left} als auch der \textit{right} Knoten als Operatoren Joins verwenden. Ist dies der Fall und überlappt die rechten Äquivalenzklassen des \textit{left} und \textit{right}-Knoten, dann kann die Regel ausgeführt werden. Vorausgesetzt eine zuvorige Prüfung mit der Methode \texttt{is\-Exchange\-Applicable()} war erfolgreich. 

\section{Enummeration}

Auch die zuvor besprochenen Regeln müssen ausgeführt und angewendet werden. Wie auch bei Pyro(J) wurde ein Button-up-Ansatz gewählt. Bei diesem Verfahren wird sich von unten nach oben durch den Baum gearbeitet. Erst wenn alle Regeln auf alle Sub-Bäume angewendet wurden, können die Regeln auch auf den Baum selbst angewendet werden. Die Komponente, die für das Ausführen und die Reihenfolge der Anwendung der Regel verantwortlich ist, ist der Enumerator. Für die hier vorgestellte Implementierung wurde ein Enumerator implementiert.


\subsection{Implementierung des Enumerators}

Bevor der Enumerator gestartet werden kann, muss er mit einer Regelmenge initialisiert werden. Die Regelmenge muss dabei dem Interface \texttt{RuleSet} entsprechen. Gestartet wird der Enumerator durch die Methdoe \texttt{apply(EquivalenceClass \&)}. Der Methode muss eine bisher noch nicht expandierte Äquivalenzklasse übergeben werden. 

Für jeden Planknoten der initialen Äquivalenzklasse wird die linke und rechte Äquivalenzklasse betrachtet. Wenn beide Klassen vorhanden sind, wird die Methode \texttt{apply(EquivalenceClass \&)} für die vorhandenen Äquivalenzknoten ausgeführt werden. Somit wird sichergestellt, dass die Regeln zuerst auf die untergeordneten Planknoten ausgeführt werden. Sobald die untergeordneten Pläne expandiert sind, wird die Regel ausgeführt. Da jede Regel neue Planknoten und neue Äquivalenzklassen hervorbringen kann, wird geprüft, ob bereits Äquivalenzklassen bekannt sind, die die selben Relationen repräsentieren. Lassen sich solche Knoten finden, werden die bereits bekannten statt der neu generierte Äuqivalenzklasse verwendet. Falls ein tatsächlich unbekannter Knoten erzeugt wurde, wird zuerst auf diese Äuqivalenzklasse die Methode \texttt{apply} angewendet und diese Äuqivalenzklasse als bekannt gespeichert. Mit hilfe dieser Methode wird verhindert, dass Berechnungen für Äquivalenzklassen mehrfach durchgeführt werden müssen. Die genaue Funktionalität kann mit Hilfe des folgenden Pseudocodes nachvollzogen werden:

\begin{algorithm}[ht]
\SetAlgoLined
\SetKwFunction{apply}{apply}
\SetKwProg{myalg}{Algorithm}{}{}

\myalg{\apply{EquivalenceClass eq}}{
    isNewEquivalenceClassSpotted = true
    
    \While{isNewEquivalenceClassSpotted}{
    
        isNewEquivalenceClassSpotted = false
        
        \For{PlanNode p in eq.PlanNodes}{
            
            \If{p.hasLeft()}
            {
                apply(p.getLeft())
            }
            
            \If{p.hasRight()}
            {
                apply(p.getRight())
            }
            
            \For{Rule r in rules}{
                \If{r.isApplicable(p)}{
                    PlanNode newPlan = r.apply(p)
                    
                    \If{isEquivalenceClassKnown(planNode.getLeft())}{
                        planNode.setLeft(getKnownClass(planNode.getLeft())
                    }
                    \Else
                    {
                        apply(planNode.getLeft())
                        isNewEquivalenceClassSpotted = true
                    }
                    
                    \If{isEquivalenceClassKnown(planNode.getRight())}{
                        planNode.setRight(getKnownClass(planNode.getRight())
                    }
                    \Else
                    {
                        apply(planNode.getRight())
                        isNewEquivalenceClassSpotted = true
                    }
                    
                    
                }
            }
            
            
        }
    }
}
\label{Pseudocod:enumerator}
\caption{Rekusiver Enumerator}
\end{algorithm}




\subsection{Erweiterbarkeit des Enumerators}

Schon das Interface des Emumerators ist erweiterbar. Andere Regelmengen können übergeben werden und so zur Ausführung kommen. Auch mit Hilfe des Template Parameters \texttt{PlanNode\_t} können die Planklassen ausgetauscht werden. Durch die Modularität der anderen Komponenten kann auch leicht ein neuer Enumerator implemenentiert werden, der den bisherigen ersetzt. Es muss kein spezielles Interface eingehalten werden. Daher muss bei der Erstellung eines neuen Enumerators nur die Exektutor Klasse verändert werden in der der bestehende Enumerator aufgerufen wird.
\section{Kostenschätzung}

Das Finden des besten Plans geschieht über die Kostenberechnung. Nur der Plan mit den niedrigsten Kosten ist der optimale Plan. Wie in System R wird davon ausgegangen, dass ein optimaler Plan aus optimalen Teilplänen besteht. Auch dieser Prototyp folgt damit dem Optimalitätsprinzip von Bellman \cite{Bellman:1957}.

Zur Ermittlung der Kosten können unterschiedliche Parameter herangezogen werden. Beispielsweise kann die CPU-Zeit, der I/O-Zugriff und andere Parameter genutzt werden. Für den Prototypen wurde eine möglichst einfache Form der Kostenschätzung auf Basis der Kardinalitäten und Selektivitäten implementiert.

In der konkreten Implementierung wird davon ausgegangen, dass die Kosten direkt in Zusammenhang mit der Kardinalität stehen und 1:1 umgerechnet werden können. So sind die Kosten für das Lesen einer Basis-Relation mit der Kardinalität von 100 auch 100. Findet ein JOIN zwischen zwei Relationen mit einer Kardinalität von je 50 statt und einer Selektvitität von 0.1 ist die Kardinalität des Join Knotens 250 und damit die Kosten für den Join 250. Nimmt man die Kosten für das Lesen der beiden Relationen hinzu (je 50), ergeben sich die Kosten für den gesamten Teilbaum. Dieses Vorgehen setzt einen Bottom-Up-Ansatz voraus. Zuerst müssen alle Teilbäume berechnet werden, bevor der gesamte Baum berechnet werden kann.

\subsection{Implementierung der Kostenfunktion}


\begin{figure}[ht]
  \centering
  \includegraphics[scale=0.75]{04_Implementierung/00_media/ClassCostEstimation.pdf}
  \caption{Klassendiagramm: Kostenschätzung}
  \label{ClassCostEstimation}
\end{figure}



Die konkrete Implementierung der Kostenfunktion wird in der Klasse \texttt{Simple\-Cost\-Estimator} vorgenommen. In ihrem Konstrukor wird die Kardinalität und die Selektvität von Basis-Relationen und Join-Kanten übergeben. Sie stammen aus der Konfiguration und sind die Grundlage der Berechnung. Die Klasse erfüllt das Interface \texttt{Cost\-Estimator}, das die Methode \texttt{find\-Optimal\-Plan(Equivalence\-Class\_t \&)} vorgibt. Dies ist in Abb. \ref{ClassCostEstimation} zu sehen.

Die Hilfsmethode  \texttt{get\-Selectivity(Plan\-Node\_t \&)} ermittelt die Selektivität für einen Planknoten.


Die Kostenberechnung wird rekursiv durchgeführt. Wie in \ref{PseudocodeCostEstimator} zu sehen, wird für jeden Planknoten, der in der Äquivalenzklasse vorhanden ist, zuerst geprüft, ob es sich um einen \texttt{SCAN} handelt. Wie zuvor beschrieben existieren zwei Operatoren \texttt{JOIN} und \texttt{SCAN}. Es ist nicht möglich, dass einem \texttt{SCAN} ein \texttt{JOIN} untergeordnet ist. 

Somit sind die Kosten für einen \texttt{SCAN} nach unserem Kostenmodell gleich der Kardinalität des \texttt{SCAN}-Planknoten. Falls es sich um keinen \texttt{SCAN} handelt, wird geprüft, ob die linke und die rechte Äuqivalenzklasse einen optimalen Planknoten gefunden haben. Wenn das nicht der Fall ist, wird die Suche nach diesen angestoßen und somit die Rekursion gestartet. Nachdem die optimalen Planknoten für die untergeordneten Äquivalenzklassen gefunden sind, kann die Kardinalität sowie die Kosten für den aktuellen Knoten berechnet werden. Falls die Kosten niedriger sind als die bisher bekannten Kosten für den bisher optimalen Plan, wird der beste Plan neu gesetzt und die optimalen Kosten und die Kardinaltität des Äquivalenzknotens angepasst.



\begin{algorithm}[ht]
\SetAlgoLined
\SetKwFunction{findOptimalPlan}{findOptimalPlan}
\SetKwProg{myalg}{Algorithm}{}{}

\myalg{\findOptimalPlan{EquivalenceClass eq}}{
    \For{PlanNode p in eq.PlanNodes}{
        \If{p.operator == SCAN}{
            eq.best = p
            
            eq.costs = p.cardinality
            
            eq.cardinality = p.cardinality
            
            return
        }
        
        \If{p.left.best == NULL}{
            findOptimalPlan(p.left)
        }
        \If{p.right.best == NULL}{
             findOptimalPlan(p.right)
        }
        
        p.cardinality = p.left.cardinality * p.right.cardinality * p.selectivity
        
        p.cost = p.left.cost + p.right.cost + p.cardinality
        
        \If{p.cost < eq.cost OR eq.cost == NULL}{
            eq.cost = p.cost
            
            eq.cardinality = p.cardinality
            
            eq.best = p
        }
    }
}
\label{PseudocodeCostEstimator}
\caption{Kostenfunktion: SimpleCostEstimator}
\end{algorithm}


\subsection{Erweiterbarkeit der Kostenfunktion}
Diese Implementierung ist nur eine starke Vereinfachung. Sie bezieht nur sehr einfache Parameter in die Berechnung ein. Keine komplexen Kostenfunktionen wurden implementiert. Dank der Modularität des Prototypen ist es möglich, neue, akkuratere Kostenfunktionen zu erstellen und diese auch zu implementieren. Dafür muss ausschließlich das Interface \texttt{Cost\-Estimator} erfüllt werden, das bereits in Abb. \ref{ClassCostEstimation} vorgestellt wurde.







%\section{Komponenten Optimierer Prototypen}




\begin{figure}[ht]
  \centering
  \includegraphics[width=\textwidth]{04_Implementierung/00_media/Organization.pdf}
  \caption{Projektorganisation}
  \label{ProjectOrga}
\end{figure}


Das Projekt ist entlang der unterschiedlichen Aufgaben modular organisiert. Unter dem Dach der Anwendung finden sich vier unterschiedliche Sparten, die gemeinsam für das Ausführen des Programms verantwortlich sind: Planknoten und Äuqivalenzklassen, Regeln und Regelmengen, Ineratoren und Services. Die genaue Zuordnung zu den jeweiligen Bereichen ist in Abb. \ref{ProjectOrga} nachvollziehbar.




Bei der Ausführung eines Tests  müssen alle Bereiche zusammenarbeiten. Der Service-Bereich und insbesondere das Konfigurationsmodul sind für die konkrete Zusammenstellung der für den Test notwendigen Komponenten verantwortlich. Er entscheidet, welche Regeln zum Einsatz kommen, welcher Enumerator gewählt wird, welche konkreten Pläne getestet und mit Hilfe welcher Adapter die Daten ein-  bzw. ausgegeben werden. Um zu verstehen, welche Möglichkeiten das implementierte System bietet, sind die einzelnen Komponenten im Folgenden im Detail beschrieben.




\subsection{Adapter}

Die Implementierung bietet mehrere Adapter, die zur Umwandlung von externen Formaten in eine interne Repräsentationsform, oder von einer internen Repräsentationsform in ein externes Format genutzt werden können. Sie bieten die Möglichkeit andere Systeme an das bestehende System anzudocken und sorgen so für den notwendigen Anschluss und die Erweiterbarkeit durch Dritte.

Insgesamt werden drei Adapter mitgeliefert. (1) JSON\-Adapter, (2) String\-Adapter, (3) DOT\-Adapter.

Der Json-Adapter erlaubt es Daten im JSON Format zu importieren und wird beim Einlesen des initalen Plans genutzt. Er wandelt auf der Basis des Konfigurationsfiles JSON in Planknoten und Äquivalenzklassen um, die dann weiterverarbeitet werden können. Ebenfalls ist es möglich, Pläne in JSON auszugeben. Zu diesem Zweck implementiert der Parser auch eine \texttt{dump} Methode.

Neben dem JSON\-Adapter wird auch ein String Adapter verwendet. Er ist für die Ausgabe von Plänen als String verantwortlich. Im Gegensatz zu einem JSON\-Adapter ist die Eingabe von Plänen mit Hilfe dieses Moduls nicht möglich. Auch der DOT\-Adapter erlaubt nur die Ausgabe von Plänen im DOT\-Format, die  zur Generierung von graphischen Ausgaben verwendet werden können.

Eine weitere Aufgabe eines solchen Adapters kann auch die Übersetzung von Relationsnamen in Bitvektoren sein. Da das vorliegende System, wie in \ref{sec:Bitvector} beschrieben, Relationen als Bitvektoren abbildet, mag es nützlich sein, Relationsnamen in Bitvektoren zu übersetzen. Für diese Übersetzung sind auch Adapter vorgesehen, die zusätzlich implementiert werden können.












\subsubsection{Erweiterbarkeit von Regelsets}





\subsubsection{Ausführung von Regeln}

Das eigentliche Ausführen der Regeln wird durch einen XY durchgeführt. Im konkreten Fall kommt hier der Algorithmus ExhaustiveTransformation zum Einsatz. Der Algorithmus startet mit einer Äquivalenzklasse. Innerhalb dieser Äquivalenzklasse werden die Regeln, die durch ein Regelset vorgegeben sind auf einem PlanKnoten ausgeführt. Die Ausführung geschieht hierbei zuerst auf den Oberen Ebenen und setzt sich dann auf den Kindern eines Knoten fort. Somit können bei der Transformation eines gegebenen Baums alle Regeln auf andere Bäume angewendet werden.

Wichtig ist hierbei zu bemerken, dass dieser Algorithmus immer zuerst prüft, ob eine Regel auch tatsächlich für die Anwendung geeignet ist und dann erst der Algorithmus ausgeführt wird. Neben der eigentlichen Eignung wird auch geprüft, ob eine Äquivalenzklasse bereits vollständig expandiert wurde. Falls dies der Fall ist, wird von einer weiteren Anwendung von Regeln abgesehen. Diese Funktion kann insbesondere Vorteile bei der Implementierung von neuen Regelsets bieten. Nutzt ein gegebenes Regelset die Möglichkeit nicht nur einen neuen Planknoten zu generieren, sondern gleich mehrere Planknoten zu erstellen und auch in diesem Zusammenhang bereits mehrere Kinder-Knoten zu erstellen, kann die Reihenfolge der Expansion von Äuqivalenzklassen geändert werden. Die einzelnen Äquivalenzklassen, die bereits durch eine Regel expandiert wurden, werden als solche markiert und die bisher vorhandenen Regeln werden nicht mehr ausgeführt.



















\subsection{Kostenschätzung und statistische Informationen}



Die Kostenschätzung zur Suche des optimalen Plans geschieht bei der vorliegenden Implementierung in einem eigenen Kostenschätzungsmodul.Teil des Kostenmoduls ist ein Katalog innerhalb dessen Informationen über Selektivität von JOINs und Kardinalitäten von Relationen gespeichert sind. Diese Informationen werden aus dem Katalog mit Hilfe des Kostenschätzers ausgelesen und die Kosten für einen Planknoten berechnet.

Die Berechnung der Kardinalität findet bottom up statt. Für jede Äquivalenzklasse wird die optimale Kardinalität berechnet und gespeichert. Der Planknoten mit dessen Hilfe diese optimale Kardinalität erreicht wird, wird als bester Planknoten markiert. Folgt man dem Pfad der besten Planknoten ergibt sich der kostenoptimale Baum an Planknoten. Die Berechnung sieht vor, dass immer die Selektivität eines Operators mit dem Produkt der Kardinalität der Eingabeparameter multipliziert wird. Beispielsweise wird auf der untersten Ebene - der Ebene der Scans - für eine gegebene Relation die Kardinalität mit der Selektivität von 1 multipliziert, da die gesamte Relation verarbeitet werden muss. Bei einem Join können komplexere Situationen auftreten. Zuerst wird das Produkt der Kardinalität der Eingabe Parameter berechnet. Diese kann dann mit der 
Selektivität multipliziert werden. Das Produkt der Kardinalitäten repräsentiert in diesem Falle die Kardinalität des möglichen Kreuzproduktes, das mit Hilfe von Selektoren eingeschränkt wird.




%\subsection{Unterschiede zu Volcano und Pyro(J)}

%Einer der fundamentalen Unterschiede zwischen Pyro bzw. Volcano zur hier vorgestellten Implementierung ist, dass sowohl Volcano als auch Pyro vollständige Optimierer sind. Sie erstellen basierend auf einer Anfrage einen physischen Plan, der dann weiterverarbeitet werden kann. Im Gegensatz dazu wird für diese Masterarbeit nur das Anpassen der Join-Reihenfolge betrachtet und daher auch nur das Anpassen der Join-Reihenfolge implementiert.


%Ebenfalls wird von Volcano und Pyro immer ein physischer Plan erzeugt. Dies geschieht in dieser Implementierung nicht. Es werden somit nur für die logischen Pläne Alternativen gefunden und aus diesen Alternativen der günstigste Plan ausgewählt. Dies geschieht, da für die Überprüfung der Regelsets RS-B0, RS-B1, RS-B2 und GraphRule keine pysischen Pläne notwendig sind. Die unterschiedlichen Regelmengen widmen sich nicht dem Berechnen von physischen Alternativen, sondern dem Finden alternativer Join-Reihenfolgen. Falls alle Pläne gefunden werden, wäre der Aufwand für die Umwandlung in physische Pläne für alle Regelmengen gleich. Daher trägt die Umwandlung und weitere Expansion nicht zu Unterschieden in der Expansionsgeschwindigkeit bei.

%Ein weiterer Unterschied zu Volcano ist, dass alle Pläne direkt berechnet werden und  anschließend aus allen Plänen der günstigste Plan ausgewählt wird. Volcano berechnet zuerst einen Plan und wählt dann aus der Menge der physischen Pläne den günstigsten aus, bevor der nächste logische Plan berechnet wird. Nur falls ein günstigerer Plan gefunden wird, wird dieser auch im Speicher behalten. Im Gegensatz zu diesem sehr ressourcensparenden Verfahren setzt Pyro und die eigen Implementierung auf Pläne die dauerhaft vorgehalten werden. Dies erleichtert das Debugging, da alle Pläne jederzeit betrachtet werden können, erhöht aber den Verbrauch an Arbeitsspeicher.

%Neben diesen konzeptionellen Unterschieden setzt die implementierte Lösung wie Pyro oder Volcano auf C++ als Programmiersprache). Im Gegensatz zu diesen Implementierungen setzt PyroJ auf Java und die Java Plattform, die per se mit schlecht beeinflussbaren Faktoren wie Garbage Collection, Virtuellen Maschinen und JIT-Compilern zu kämpfen hat. Wie bereits in Kapitel \ref{} besprochen, mussten bei den Messungen mit PyroJ diverse Parameter gesetzt werden, um überhaupt reproduzierbare Ergebnisse zu erzeugen. Um solchen Problemen vorzubeugen, wurde vollständig auf C++ für die Implementierung gesetzt. Da der Code zur Laufzeit bereits vollständig kompiliert ist, können Probleme durch einen JIT Compiler und Optimierungen zur Laufzeit ausgeschlossen werden. Unterbrechungen durch eine Garbage Collection können nicht auftreten, da keine vorhanden sind.
