\chapter{Resultate und Evaluation}

Nachdem die Grundlagen besprochen  und ein Prototyp implementiert worden ist, wurden unterschiedliche Experimente durchgeführt und die bisherige Arbeit kritisch reflektiert. In diesem Kapitel werden die Ergebnisse auf unterschiedlichen Feldern beleuchtet: Zuerst wird auf die Regelmengen eingegangen. Es wird die Unvollständigkeit von RS-B2 als Anlass für eine genaue Prüfung genommen. Zuerst werden die in Kapitel 3 aufgestellten Beispiele für Unvollständigkeit von sternförmigen Abfragen mit eigenen Messungen überprüft. Ebenfalls wird auf das theoretische Beispiel, das als Beleg für die Unvollständigkeit herhalten musste mit eigenen Mitteln nachstellt.

Auf Basis der Ergebnisse werden Indizien gesucht, in welchen Fällen RS-B2 unvollständig ist. Hierzu wird zuerst ein morphologischer Kasten aufgestellt. Aus ihm werden Test gebildet und besprochen.


Die Geschwindigkeit der Regelmengen wurde in 5.2 verglichen. Hier zeigt sich, dass die Geschwindigkeit erheblich von einem passenden Enumerator bzw. Expander abhängt. GraphRule bzw. MinCutConservative ist, wie zu erwarten, in allen Fällen am schnellsten.

Nachdem diese Messergebnisse besprochen wurden, wird noch auf die Wahl von Pyro(J) als Testplattform eingegangen. Die Plattform war die Grundlage für die in Kapitel 3 vorgestellten Test. Wie in 5.1 festgestellt wurde, kann die Plattform auch der Grund für die z.T. gemessene Unvollständigkeit von RS-B2 sein.




\section{Gemessene Unvollständigkeit von RS-B2-CPS und Pyro(J)}

Wie in Kapitel 3 gesehen, wurde die Unvollständigkeit von \cite{bachelor}  mit Hilfe einer sternförmigen Anfrage nachgewiesen. Diese Ergebnisse sollten reproduziert werden. Da diese Ergebnisse so nicht reproduzierbar waren, wird ein Erklärungsansatz vorgeschlagen.

\subsection{Prüfung der Ergebnisse aus Kapitel 3}
Es wurden zwei Tests durchgeführt. Hierzu wurden zwei sternförmige Anfragen erzeugt. Jede Anfrage wurde ausgeführt und die Anzahl der unterschiedlichen Pläne gemessen.

\begin{figure}[h]
\centering

\begin{tabular}{|l|l|l|}
\hline
Relationen & RS-B2 & RS-B1 \\ \hline
5          & 384       & 384       \\ \hline
7          & 46080     & 46080     \\ \hline
\end{tabular}





\caption{Anzahl Pläne nach Regelmenge und Anzahl-Relationen}
\label{protoUnvollstaendig}
\endfigure

Im Gegensatz zu \cite{bachelor} konnte nicht festgestellt werden, dass RS-B2 schlechter als die Vergleichsregelmenge RS-B1 funktioniert.


\subsection{Erklärungsansatz: Pyros Expansionsalgorithmus}

Der Grund für die Abweichung mag am Expansionsalgorithmus in Pyro liegen. Der verwendete Expansionsalgorithmus (vgl. Abb. \ref{ExpandDAG}) ist vollständig für die Regelmenge RS-B1. Ob dies auch für die Regelmenge RS-B2 zutrifft, wird im Folgenden an einem konkreten Beispiel geprüft.

\begin{figure}[ht]
  \centering
  \includegraphics[scale=0.75]{05_ResultsEvaluation/00_media/PyroInital_0.pdf}
  \caption{Initialer Query Tree}
  \label{PyroInital}
\end{figure}

\begin{figure}[ht]
  \centering
  \includegraphics[scale=0.75]{05_ResultsEvaluation/00_media/PyroJoinGraph.pdf}
  \caption{Join-Graph}
  \label{PyroJoinGraph}
\end{figure}

Die Prüfung des Algorithmus erfolgt in vereinfachter Form am Beispiel des in Abb. \ref{PyroInital} dargestellten Query-Tree und Join-Graphen aus Abb. \ref{PyroJoinGraph}. Um die Darstellung zu vereinfachen, wird die Regel Kommutativität der Regelmenge nicht angewendet. Es bleiben also linke-Assoziativität, rechte-Assoziativität und Exchange erhalten.

In einem ersten Schritt sollen die Regeln auf Planknoten 1 angewendet werden. Da die untergeordneten Planknoten noch nicht expandiert wurden, muss die Expansion auf die Knoten 2 bzw. 3 angewendet werden. Da sich auf Knoten 3 keine Regel anwenden lässt, wird dieser Knoten als expandiert markiert. Auf den Knoten 2 kann Assoziativität angewendet werden. Es entsteht so der neue Knoten 4. Auf Knoten 4 lässt sich keine Regel anwenden. Alle Regeln wurden für Knoten 2 angewendet, daher kann Knoten 2 als expandiert markiert werden.

Da nun alle untergeordneten Knoten expandiert sind, können auf der Ebene von Knoten 1 alle Regeln angewendet werden. Hier ist die Anwendung von Exchange am interessantesten. Es werden die neuen Knoten 7, 8, 9, 10, 11 geschaffen. Da auf die Knoten 7, 9 keine weiteren Regeln angewendet werden dürfen, bricht der Algorithmus ab. 

\begin{figure}[ht]
  \centering
  \includegraphics[scale=0.75]{05_ResultsEvaluation/00_media/PyroResult.pdf}
  \caption{Expandierter Query Tree}
  \label{ExpandedQueryTree}
\end{figure}

Wie in Abb. \ref{ExpandedQueryTree} zu erkennen, sind die Regeln nicht auf alle Teilbäume angewendet worden. Der gesamte Suchraum wurde nicht erforscht. Einer der Unterschiede zwischen dem von Pyro verwendeten und dem in Abb. \ref{ExpandDAG} gezeigten Expander ist, dass sobald ein neuer Planknoten gefunden wurde versucht wird alle Regeln, solange es möglich ist auf den gesamten Teilbaum erneut anzuwenden. So werden auch neue, untergeordnete Teilbäume in die Expansion einbezogen und der gesamte Suchraum im konkreten Beispiel erforscht.


\subsection{Prüfung der theoretischen Unvollständigkeit}
In Kapitel 3 wurde ein spezieller Fall vorgestellt, in dem die Unvollständigkeit der Regelmenge RS-B2 gezeigt wurde. Auch dieses Beispiel wurde geprüft.

Die Genaue Konfiguration kann entweder in Abbildung 3.1 bzw. Abbildung \ref{configUnvollstae} gefunden werden.

\begin{figure}[ht]
  \centering
  \includegraphics[width=\textwidth]{05_ResultsEvaluation/00_media/Vollstaendigkeitstest.png}
  \caption{Konfiguration: Vollständigkeitstest}
  \label{configUnvollstae}
\end{figure}

Bei den Tests zeigt sich (vgl. Abb. \ref{Result:Unvollstaendigkeit}), dass trotz der Wahl des akkuraten Transformatuonsenumerators korrekte Anzahl der Pläne mit RS-B2 nicht erreicht werden konnte. Die Menge der gefundenen Pläne liegt bei nur knapp 50\% der Pläne, die eigentlich zu erwarten sind. Auch die Anzahl der generierten Äquivalenzklassen ist stark reduziert. Durch diese Reduktion an Knoten konnte jedoch auch der Aufwand für die Enumeration um fast 50\% gesenkt werden.

\begin{figure}[ht]
\centering
\begin{tabular}{|l|l|l|l|}
\hline
                        & {\bf RS-B0} & {\bf RS-B1} & {\bf RS-B2} \\ \hline
{\bf Äuqivalenzklassen} & 101         & 101         & 85          \\ \hline
{\bf Pläne}             & 10.752      & 10.752      & 5.760       \\ \hline
{\bf Dauer (ms)}        & 2.764.217   & 2.620.888   & 1.378.230   \\ \hline
\end{tabular}
\caption{Resultat: Vollständigkeitstest}
  \label{Result:Unvollstaendigkeit}
\end{figure}

Die Unvollständigkeit der Regelmenge RS-B2 ist mit diesem Test erneut belegt. In der nächsten Sektion soll näher untersucht werden, in welchen Fällen die Regelmenge unvollständig ist.



\section{Evaluation der Vollständigkeit}

In einem ersten Schritt wird auf die Vollständigkeit der Pellenkoft Regelmengen eingegangen. Im Folgenden wird dann die Vollständigkeit von RS-Rule genauer betrachtet.

\subsection{Vollständigkeit von RS-B0, RS-B1 und RS-B2}

Mit Hilfe des implementierten Optimierers können unterschiedliche Regelmengen ausgeführt werden. Wie in Kapitel 4 beschrieben, lassen sich neue Regeln leicht hinzufügen, Testszenarien generieren und die Ergebnisse speichern. Um die Vollständigkeit der Regelmengen zu prüfen, wurden unterschiedliche Testszenarien generiert und die einzelnen Regelmengen auf ihre Ausgabe hin überprüft. In einem ersten Schritt musste die Korrektheit der Implementierung geprüft werden.

Die Tests wurden mit Hilfe des folgenden morphologischen Kastens gebildet und ausgeführt. So konnten unterschiedliche Test Arten geprüft werden:

\begin{figure}[ht]
\centering
\resizebox{\textwidth}{!}{%
\begin{tabular}{|l|l|l|l|l|l|}
\hline
{\bf Algorithmus} & {\bf Anzahl an Knoten} & {\bf Formen} &  {\bf Initialer Baum} & {\bf Selektvität}    & {\bf Kardinalitäten} \\ \hline
RS-B0             & 5                      & Stern        & Bushy                 & Standard (0.5)       & Standard (10)        \\ \hline
RS-B1             & 7                      & Kette        & Linkstief             & Zufällig (0.0 - 1.0) & Zufällig (1 -100)    \\ \hline
RS-B2             & 10                     & Baum         & Rechtstief            &                      &                      \\ \hline
GraphRule         & 12                     & Zyklisch     &                       &                      &                      \\ \hline
\end{tabular}
}
\caption{Morthologischer Kasten zur Testgenerierung}
\label{my-label}
\end{figure}

Für jeden Test wurde das exakte Testszenario, Beobachtungen und Ergebnisse dokumentiert, um sie besser nachvollziehbar zu machen.



\newtheorem{hypo}{Hypothese} 





Einige Tests wurden ausgeführt, um die Performance und Vollständigkeit der einzelnen Regelmengen zu prüfen. Entlang der Dimensionen Regelmenge, Anzahl der Relationen, Form des Join Graphen, Form des initialen Anfragebaums, Selektivität und Kardinalität wurde ein morphologischer Kasten aufgestellt. Auf dessen Basis konnten verschiedene Tests behandelt werden. 

\subsection{Vollständigkeit der Regelmengen bei azyklischen linkstiefen Bäumen }

\begin{hypo}
Alle Regelmengen sind bei azyklischen Join-Bäumen und linkstiefen initialen-Bäumen vollständig.
\end{hypo}

\begin{figure}[ht]
  \centering
  \includegraphics[width=\textwidth]{05_ResultsEvaluation/00_media/Test1.png}
  \caption{Konfiguration: Test 1}
  \label{Konfiguration:Test1}
\end{figure}

Zu diesem Zweck wurde eine Konfiguration mit fünf Relationen gewählt, die wie in Abbildung \ref{Konfiguration:Test1} zu sehen mit einander verknüpft sind.

Um festzustellen, ob alle Algorithmen für diesen speziellen Fall korrekt funktionieren, wurden zuerst alle erzeugten Pläne ausgegeben und verglichen. In einem zweiten Schritt wurden alle Äquivalenzklassen betrachtet und auf ihren Inhalt manuell geprüft.



\begin{figure}[ht]
\centering

\begin{tabular}{|l|l|l|l|}
\hline
                         & \multicolumn{3}{c|}{{\bf Result}} \\ \hline
{\bf Algorithmus}        & RS-B0     & RS-B1     & RS-B2     \\ \hline
{\bf Anzahl Bäume}       & 224       & 224       & 224       \\ \hline
{\bf Dauer}              & 28558  & 25523  & 23280  \\ \hline
\end{tabular}

\caption{Resultat: Test 1}
\label{Resultat:Test1}
\end{figure}

Wie in Tabelle \ref{Resultat:Test1} zu sehen ist, wurden je 224 unterschiedliche Bäume erzeugt. Eine Prüfung ergab, dass die Pläne, die durch die drei Regelmengen erzeugt wurden, alle gleich sind.

In einem nächsten Schritt wurden die Äuqivalenzklassen auf ihren Inhalt geprüft. In der obersten Äquivalenzklasse fanden sich die folgenden Planknoten:

\begin{itemize}
\item join(1,2345)
\item join(12,345)
\item join(123,45)
\item join(1234,5)
\item join(2345,1)
\item join(345,12)
\item join(45,123)
\item join(5,1234)
\end{itemize}

Bei diesen Planknoten handelt es sich um alle möglichen Permutationen der Äquivalenzklasse, die die Relationen 1 bis 5 repräsentiert. Auch alle anderen 34 Äquivalenzklassen wurden manuell geprüft. Das Ergebnis ist klar: Alle Permutationen wurden generiert und somit alle Pläne gefunden.

Es lässt sich also feststellen, dass alle Regelmengen in Bezug auf linkstiefe, azyklische Query Graphen mit 5 Relationen vollständig sind.

\subsection{Prüfung von azyklischen, sternförmigen Anfragen}

In einem zweiten Schritt wurde die Aussage von \cite{shanbhag2014optimizing} als Grundlage genommen, der behauptete, dass sternförmige Join-Trees zu umvollständigen Resultaten führen würden.


\begin{hypo}
Auf Basis eines azyklischen, sternförmigen Join-Trees lässt sich mit RS-B2 der Suchraum nicht vollständig erforschen.
\end{hypo}


\begin{figure}[ht]
  \centering
  \includegraphics[width=\textwidth]{05_ResultsEvaluation/00_media/Test2.png}
  \caption{Konfiguration: Test 2}
  \label{Konfiguration:Test2}
\end{figure}


Um die Hypothese zu untersuchen, wurde eine Konfiguration gewählt, die der aus Test 1 gleicht. Einziger Unterschied sind andere Join Kanten. Wie in  Abb. \ref{Konfiguration:Test2} zu sehen, sind hier alle Knoten per Join mit Knoten 1 verbunden. Eine Sternform ergibt sich.

Nach den vorhergehenden Tests von \cite{shanbhag2014optimizing} wurde erwartet, dass durch RS-B2 nicht alle Pläne gefunden werden würden. Wie in Tabelle \ref{Result:Test2} zu sehen ist, wurden jedoch genau gleich viele Pläne mit Hilfe RS-B2 wie mit RS-B1 gefunden. 

\begin{figure}[ht]
\centering

\begin{tabular}{|l|l|l|l|}
\hline
                         & \multicolumn{3}{c|}{{\bf Result}} \\ \hline
{\bf Algorithmus}        & RS-B0     & RS-B1     & RS-B2     \\ \hline
{\bf Anzahl Bäume}       & 384       & 384       & 384       \\ \hline
{\bf Dauer}              & 66643  & 65110  & 58225   \\ \hline
\end{tabular}

\caption{Resultat: Test 2}
\label{Result:Test2}
\end{figure}


Auch die Pläne aus Test 2 wurden verglichen. Pro Regelmenge wurden die gleichen Pläne generiert. Alle 43 Äquivalenzklassen wurden geprüft und ihre Vollständigkeit festgestellt.

\begin{figure}[ht]
  \centering
  \includegraphics[width=\textwidth]{05_ResultsEvaluation/00_media/Test3.png}
  \caption{Konfiguration: Test 3}
  \label{Konfiguration:Test3}
\end{figure}


Um sicherzustellen, dass die Aussage von \cite{shanbhag2014optimizing} nicht abhängig von der Form des Baums ist, wurde ein weiterer Test durchgeführt. Dieses Mal wurde ein recht-tiefer Baum gewählt. Die Konfiguration (vgl. Abb. \ref{Konfiguration:Test3}) gleicht der vorhergehenden, ausschließlich die Join-Reihenfolge wurde verändert.

\begin{figure}[ht]
\centering
\begin{tabular}{|l|l|l|l|}
\hline
                         & \multicolumn{3}{c|}{{\bf Result}} \\ \hline
{\bf Algorithmus}        & RS-B0     & RS-B1     & RS-B2     \\ \hline
{\bf Anzahl Bäume}       & 384       & 384       & 384       \\ \hline
{\bf Dauer}              & 66643  & 65110  & 58225  \\ \hline
\end{tabular}

\caption{Resultat: Test 3}
\label{Result:Test3}
\end{figure}


Wie schon im vorhergehenden Test wurden alle Pläne geprüft. Es wurden gleich viele und die gleichen Pläne wie im vorhergehenden Fall ausgegeben. Daher kann davon ausgegangen werden, dass bei einer Menge von 5 Relationen für die Vollständigkeit nicht entscheidend ist, ob es sich initial um einen linkstiefen oder einen rechtstiefen Baum gehandelt hat.



\subsection{RS-B2 ist vollständig bei zyklischen Graphen}
Um dennoch Unvollständigkeit zu finden, wurde eine neue Hypothese aufgestellt:

Hypothese 3: RS-B2 ist bei zyklischen Join Graphen unvollständig.

Um diese Hypothese zu prüfen, wurde Test 3 herangezogen und eine neue Join-Kante zwischen 3 und 4 eingezogen (vgl. Abb. \ref{Konfiguration:Test3}. Durch diese zusätzliche Kante ist ein zyklischer Join-Graph entstanden. 


\begin{figure}[ht]
  \centering
  \includegraphics[width=\textwidth]{05_ResultsEvaluation/00_media/Test4.png}
  \caption{Konfiguration: Test 4}
  \label{Konfiguration:Test4}
\end{figure}

Wie erwartet, wurden weniger (ca. 50\%) Pläne gefunden (vgl. Tabelle \ref{Result:Test4}). Bei einem Vergleich fiel auf, dass gerade entlang der zyklischen-Join-Kanten weniger Pläne generiert wurden.


\begin{figure}[h]
\centering
\begin{tabular}{|l|l|l|l|}
\hline
                         & \multicolumn{3}{c|}{{\bf Result}} \\ \hline
{\bf Algorithmus}        & RS-B0     & RS-B1     & RS-B2     \\ \hline
{\bf Anzahl Bäume}       & 480       & 480       & 272       \\ \hline
{\bf Dauer}              & 60197027  & 56166774  & 28628930  \\ \hline
\end{tabular}

\caption{Resultat: Test 4}
\label{Result:Test4}
\end{figure}

Es zeigt sich mit diesem Experiment, dass RS-B2 nicht geeignet sind, um alle Pläne für einen zyklischen Join-Graphen zu erzeugen.

\subsection{RS-B2 ist Vollständig für bestimmte zyklische Join-Graphen}







\subsection{RS-B2-CPS Vollständigkeit bzgl. sternförmigen Anfragen}

Wie in Kapitel 3 gesehen wurde die Unvollständigkeit von \cite{bachelor}  mit Hilfe einer sternförmigen Anfrage nachgewiesen. Dieses Ergebnisse sollte reproduziert werden. Da diese Ergebnisse so nicht reproduzierbar waren, wird ein Erklärungsansatz vorgeschlagen.

\subsubsection{Prüfung der Ergebnisse aus Kapitel 3}
Es wurden zwei Tests durchgeführt. hierzu wurden zwei sternförmige Anfragen erzeugt. Jede Anfrage wurde ausgeführt und die Anzahl der unterschiedlichen Pläne gemessen.

\begin{figure}[h]
\centering

\begin{tabular}{|l|l|l|}
\hline
Relationen & RS-B2-CPS & RS-B1-CPS \\ \hline
5          & 284       & 284       \\ \hline
7          & 46080     & 46080     \\ \hline
\end{tabular}





\caption{Anzahl Pläne nach Regelmenge und Anzahl-Relationen}
\label{protoUnvollstaendig}
\endfigure

Im Gegensatz zu \cite{bachelor} konnte nicht festgestellt werden, dass RS-B2 schlechter als die Vergleichsregelmenge RS-B1 funktioniert.


\subsubsection{Erklärungsansatz: Pyros Expansionsalgorithmus}

Der Grund für die Abweichung mag am Expansionsalgorithmus in Pyro liegen. Der verwendete Expansionsalgorithmus (vgl. Abb. \ref{ExpandDAG}) ist vollständig für die Regelmenge RS-B1. Ob dies auch für die Regelmenge RS-B2 zutrifft, wird im folgenden an einem konkreten Beispiel geprüft.

\begin{figure}[ht]
  \centering
  \includegraphics[scale=0.75]{05_ResultsEvaluation/00_media/PyroInital_0.pdf}
  \caption{Initialer Query Tree}
  \label{PyroInital}
\end{figure}

\begin{figure}[ht]
  \centering
  \includegraphics[scale=0.75]{05_ResultsEvaluation/00_media/PyroJoinGraph.pdf}
  \caption{Join-Graph}
  \label{PyroJoinGraph}
\end{figure}

Die Prüfung des Algorithmus erfolgt in vereinfachter Form am Beispiel des in Abb. \ref{PyroInital} dargestellten Query-Tree und Join-Graphen aus Abb. \ref{PyroJoinGraph}. Um die Darstellung zu vereinfachen, wird die Regel Kommutativität der Regelmenge nicht angewendet. Es bleiben also linke-Assoziativität, rechte-Assoziativität und Exchange erhalten.

In einem ersten Schritt sollen die Regeln auf Planknoten 1 angewendet werden. Da die untergeordneten Planknoten noch nicht expandiert wurden, muss die Expansion auf die Knoten 2 bzw. 3 angewendet werden. Da sich auf Knoten 3 keine Regel anwenden lässt, wird dieser Knoten als expandiert markiert. Auf den Knoten 2 kann Assoziativität angewendet werden. Es entsteht so der neue Knoten 4. Auf Knoten 4 lässt sich keine Regel anwenden. Alle Regeln wurden für Knoten 2 angewendet, daher kann Knoten 2 als expandiert markiert werden.

Da nun alle untergeordneten Knoten expandiert sind, kann auf Ebene von Knoten 1 alle Regeln angewendet werden. Hier ist die Anwendung von Exchange am interessantesten. Es werden die neuen Knoten 7, 8, 9, 10, 11 geschaffen. Da auf die Knoten 7, 9 keine weiteren Regeln angewendet werden dürfen, bricht der Algorithmus ab. 

\begin{figure}[ht]
  \centering
  \includegraphics[scale=0.75]{05_ResultsEvaluation/00_media/PyroResult.pdf}
  \caption{Expandierter Query Tree}
  \label{ExpandedQueryTree}
\end{figure}

Wie in Abb. \ref{ExpandedQueryTree} zu erkennen, sind die Regeln nicht auf alle Teilbäume angewendet worden. Der gesamte Suchraum wurde nicht erforscht. Einer der Unterschiede zwischen dem von Pyro verwendeten und dem in Abb. \ref{Pseudocod:enumerator} gezeigten Expander ist, dass sobald ein neuer Planknoten gefunden wurde versucht wird alle Regeln, solange es möglich ist auf den gesamten Teilbaum erneut anzuwenden. So werden auch neue, untergeordnete Teilbäume in die Expansion einbezogen und der gesamte Suchraum im konkreten Beispiel erforscht.



\section{Effizienz der Regelmengen}

Um zu Prüfen, ob die neu eingeführte GraphRule tatsächlich effizienter als die bisher vorhandenen Regelmengen ist, muss die Geschwindigkeit zur Erzeugung eines Plans gemessen und verglichen werden. Die Messung der Zeit, die benötigt wird, um alle Pläne zu expandieren, wurde von \cite{} gemessen, wobei nur die Generierung des logischen Suchraums und nicht die Generierung des physischen Suchraums einbezogen wurde.

Zuerst wurde die Performance der einzelnen Regelmengen in Bezug auf ketten-, stern-, kreisförmige sowie zyklische  Graphen getestet. Jeder Test wurde mit einer unterschiedlichen Anzahl an Relationen durchgeführt. Alle Anfragen bestanden nur aus Joins, damit keine weiteren Regeln Anwendung finden. Später wurden auch Test durchgeführt, die den Push-Down bzw. Pull-up von aggregierten Relationen beinhalten, um damit die Anpassungsfähigkeit von GraphRule zu prüfen.



\subsection{Implementierung von Pyro(J)}
\label{sec:pyroJ}
PyroJ ist der Optimierer, der von \cite{shanbhag2014optimizing} für die Prüfung der Vollständigkeit von Pellenkofts Regelmenge RS-B2 genutzt wurde. Dies wird im folgenden Kapitel genauer beschrieben. Ebenfalls wurde auf Basis dieses Optimierers die neue Regelmenge RS-Graph implementiert und getestet. 
PyroJ basiert auf dem von \cite{roy2001multi} in C++ implementierten Optimierer Pyro und wurde automatisch von C++ nach Java übersetzt. Der Optimierer Pyro wurde nach dem Vorbild des Volcano Optimierers entwickelt. Volcano wurde als Vorbild gewählt, da es sich bei Volcano um einen hoch-respektierten, state-of-the-art, regelbasierten Optimierer handelt, der auch die Basis von kommerziellen Datenbanksytemen wie MS SQL Server ist. Gerade die Erweiterbarkeit im Hinblick auf das Datenmodell, Executionsmodell und die Möglichkeit Transformationsregeln und Operatoren hinzuzufügen, sollten übernommen werden.


Einige Unterschiede zwischen Volcano und Pyro bestehen jedoch aus Sicht von \cite{roy2001multi}:

\subsubsection{Trennung zwischen logischem und physischem Planspace}

Pyro generiert zuerst logische Pläne, die dann in physische Pläne umgesetzt werden. In einem dritten Schritt kann daraus der optimale Plan ausgewählt werden. Diese Schritte werden nacheinander unabhängig ausgeführt. In der Realität können diese drei Schritte überlappen. So ist es möglich, dass zuerst für einen logischen Plan alle physischen Pläne erzeugt werden und dann aus diesen Plänen nur der günstigste behalten wird. Daraufhin kann dann der nächste logische Plan erzeugt und für ihn der günstigste physische Plan gesucht werden. Nur wenn dieser günstigste Plan billiger ist als der bisher gefundene Plan, wird der physische Plan auch weiterhin gespeichert. Dieser Ansatz kann Ressourcen-schonender sein, als der in Pyro verwendete Ansatz, bei dem immer alle Daten vorgehalten werden. 

\subsubsection{Vereinigung von Äuqivalenten-Subausdrücken}

Bei Volcano ist es möglich gewesen, dass mehrere Äquivalenzklassen den selben Knoten repräsentieren. Beispielsweise kommt in der Anfrage $(A \Join B \Join C) \cup (B \Join C \Join D)$ $B$ und $C$ zweimal vor. Obwohl dies der Fall ist, werden für $B$ und für $C$ je zwei Äquivalenzklassen erzeugt. Nachdem diese beiden Relationen als unabhängig betrachtet werden, fällt auch nicht auf, dass der Ausdruck $B \Join C$ mehrfach vorkommt und somit auch in einer Äquivalenzklasse behandelt werden kann. Später wurde dieses Problem bei Volcano erkannt und mit Hilfe einer Memofunktion gelöst.

Auch Pyro(J) implementiert eine solche Memofunktion, die die Wiederverwendung von bekannten Äuqivalenzklassen erlaubt.


\subsubsection{Generierung mit Description-Files}

Ein fundamentaler Unterschied zwischen Volcano und Pyro auf den von \cite{roy2001multi} nicht hingewiesen wird, ist die Erstellung des Optimierers. Bei Pyro handelt es sich um einen fertigen Optimierer, der nicht mehr generiert werden muss. Es sind keine Description-Files vorhanden. Eine einfache Konfiguration ist nicht mehr möglich. Eine Generierung eines Optimierers findet überhaupt nicht statt.


\subsubsection{Ausführung der Experimente}


\begin{figure}[ht]
  \centering
  \includegraphics[scale=0.75]{02_Related_Work/ExpandDAG.png}
  \caption{Pseudocode: ExpandDAG}
  \label{ExpandDAG}
\end{figure}

Auf PyroJ wurden von \cite{shanbhag2014optimizing} Experimente zur Überprüfung der Pellenkoft Regelmengen durchgeführt. Die Durchführung der Tests geschah mit Hilfe des Expanders \texttt{ExpandDAG}. Die Prozedur (vgl. Abb. \ref{ExpandDAG}) expandiert basierend auf einem Äquivalenzknoten unterschiedliche Transformationen aus und speichert diese in Äquivalenzklassen. Der Algorithmus, der von \cite{roy2001multi} implementiert wurde, sieht vor, dass in jeder Äquivalenzklasse über die Menge der bisher noch nicht expandierten Planknoten iteriert wird. Falls die dem Planknoten untergeordneten Äquivalenzklassen bisher noch unbekannt sind, werden diese Äquivalenzklassen expandiert. Sobald alle untergeordneten Pläne expandiert wurden, werden die Regeln auf den aktuellen Planknoten angewendet. Neue Alternativen werden der bestehenden Äquvivalenzklasse untergeordnet.
\cite{roy2001multi} legt außerdem in einem Beweis dar, dass dieser Algorithmus für RS-B1 mit Prädikatpushdown (vgl. Kapietl 3) vollständig ist und alle Pläne gefunden werden können. 

Wie im folgenden Kapitel beschrieben, wurden unterschiedliche Regelmengen auf ihre Vollständigkeit und Performance getestet. Außerdem wurde eine neue Regel implementiert. 





