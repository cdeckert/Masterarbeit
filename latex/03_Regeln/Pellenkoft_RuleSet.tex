\section{Pellenkoft Regelmengen}
\label{sec:pellenkoftRulesets}
Bei der Erforschung eines Search Spaces kommen in regelbasierten Anfrageoptimierern Transformationsregeln zum Einsatz. Pellenkoft et al. \cite{pellenkoft1997duplicate} \cite{manegold2000multi} \cite{pellenkoft1997complexity} stellt drei Regelmengen zur Verfügung, die bei der Erzeugung eines Search Spaces zum Einsatz kommen können.



Zu Beginn besteht ein Search Space aus einem Plan. Dieser Plan stellt den Startpunkt dar. Auf ihn werden Transformationsregeln angewendet. Neue Pläne entstehen. Sollten diese Pläne noch nicht bekannt sein, werden sie dem Suchraum hinzugefügt. Pläne auf die bereits alle anwendbaren Transformationsregeln angewendet wurden, werden als besuchte Pläne bezeichnet. Sobald alle Pläne besucht wurden und somit auf alle Pläne Transformationsregeln angewendet wurden, ist der Suchraum vollständig erforscht und keine weiteren Pläne können gefunden werden. Der Actual Search Space ist ausgeschöpft.

Mehrere Transformationsregeln werden gemeinsam als Regelmenge bezeichnet. Pellenkoft et al. unterscheidet zwischen Regelmengen, die Pläne mehrfach generieren können und Regelmengen, die duplikatfrei sind. Beispielsweise kann durch die Anwendung der Regel Kommutativität auf einen Plan und erneute Anwendung auf dessen Resultat  der ursprüngliche Plan generiert werden. Im Folgenden werden zwei Regelmengen vorgestellt, die Duplikate bilden und eine Regelmenge, die duplikatfrei ist.


\subsection{Regelmengen mit Duplikaten}

Eine der Regelmengen, die zur Erzeugung eines Bushy Tree Space genutzt werden kann, ist \textit{RS-B0}. Sie besteht aus drei Regeln:

\begin{itemize}
\item Kommutativität: $$ A \Join B \to B \Join A$$
\item Rechte Assoziativität: $$(A \Join B) \Join C \to A \Join (B \Join C) $$
\item Linke Assoziativität: $$A \Join (B \Join C) \to (A \Join B) \Join C$$
\end{itemize}

Die Regelmenge ist redundant, da mit Hilfe von Kommutativität und rechter Assoziativität linke Assoziativität (und vis-à-vis) erzeugt werden kann. Möchte man die Redundanz vermeiden, lässt sich die Regelmenge \textit{RS-B1} ableiten. Sie basiert auf Kommutativität und einer Swap genannten Regel:

\begin{itemize}
\item Swap $$ (A \Join B) \Join C \to (A \Join C) \Join B $$
\item Bottom Commutativitity $$ B_1 \Join B_2 \to B_2 \Join B_1$$
\end{itemize}



Durch die Anwendung der Regeln aus \textit{RS-B0} und \textit{RS-B1} können Pläne doppelt erzeugt werden. Am einfachsten ist es, dies an Hand von Kommutativität zu zeigen. Wird auf den Plan $A \Join B$ Kommutativität angewendet, entsteht $B \Join A$, dann entsteht durch die erneute Anwendung von Kommutativität auf den neuen Plan $B \Join A$ wieder der ursprüngliche Plan.

Ebenfalls können sich bei komplexeren Plänen  Teilpläne gleichen. Beispielsweise enthält der Plan $(A \Join B) \Join C$ den gleichen Subplan wie $C \Join (A \Join B)$.

Um Duplikate zu verhindern, muss während des Ausführens der Regeln darauf geachtet werden, dass Duplikate erkannt und behandelt werden. Allerdings existiert auch eine Regelmenge, die per se keine Duplikate generiert. Diese wird im Folgenden behandelt.



\subsection{Duplikatfreie Regelmenge}
Durch die Anwendung von \textit{RS-B0} bzw. \textit{RS-B1} ist es möglich, dass Varianten des Plans erneut erzeugt werden. Dieser Gefahr trägt die Regelmenge \textit{RS-B2} Rechnung. Sie sieht vor, dass eine Regel nur genau einmal ausgeführt und andere Regeln nur einmal pro Operator ausgeführt werden dürfen. Dieses Regelmenge besteht aus:


\begin{itemize}
\item Kommutativität: $$ A \Join_1 B \to B \Join_2 A$$
Alle Regeln sind für den Operator $\Join_2$ deaktiviert.
\item Rechte Assoziativität: $$(A \Join_1 B) \Join_2 C \to A \Join_3 (B \Join_4 C) $$
Alle Regeln außer Kommutativität sind für $\Join_3$ deaktiviert.
\item Linke Assoziativität: $$A \Join_1 (B \Join_2 C) \to (A \Join_3 B) \Join_4 C$$
Alle Regeln außer Kommutativität sind für $\Join_4$ deaktiviert.

\item Exchange: $$(A \Join_1 B) \Join_2 (C \Join_3 D) \to (A \Join_4 D) \Join_5 (C \Join_6 B) $$
Alle Regeln außer Kommutativität sind für $\Join_5$ deaktiviert.
\end{itemize}

Regeln, die für einen Operator deaktiviert wurden, dürfen für diesen Operator nicht mehr ausgeführt werden.


Die Duplikatfreiheit wird dadurch erreicht, dass beispielsweise Kommutativität nur einmal auf einen Knoten angewendet werden kann. Knoten, die aus dieser Anwendung entstehen, dürfen Kommutativität nicht mehr anwenden. Da nach der Kommutativität keine weiteren Regeln auf einen Knoten angewendet werden dürfen, ist nach der Ausführung der Kommutativität klar, dass der ursprüngliche Plan nicht mehr erreicht werden kann. Bei den anderen Regeln ist es erlaubt, dass Kommutativität noch zur Anwendung kommt. Dies erlaubt es jedoch ebenfalls nicht den Ursprungszustand wieder zu erreichen. Somit ist es nicht möglich, dass Duplikate entstehen. Somit ist die Regelmenge duplikatfrei.



