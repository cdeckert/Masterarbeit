\section{GraphRule und Unvollständigkeit von RS-B2}

Die von Pellenkoft et al. entwickelten Regelsets, dienen als Grundlage für regelbasierte Optimierer wie Volcano. \cite{shanbhag2014optimizing} untersuchten die Vollständigkeit der vorgestellten Regel und fügten den bestehenden Regelsets etwas Neues hinzu, das alle andere Regelsets zum Join Reordering obsolet macht und zudem eine erheblich bessere Performance liefert.

Die Untersuchungen \cite{shanbhag2014optimizing} wurden mit dem Optimierer PyroJ durchgeführt. PyroJ ist ein Optimierer, der basierend auf Pyro \cite{roy2001multi} erstellt wurde und dem Volcano Optimizer nachempfunden ist. Zuerst wurden die Regelsets auf ihre Vollständigkeit hin geprüft. Neben der Prüfung auf Vollständigkeit, konnte auch die neue Regel GraphRule implementiert und so bzgl. ihrer Performance getestet werden.



\subsection{Getestete Regeln und Regelsets}

\cite{shanbhag2014optimizing} und \cite{roy2001multi} geben an, dass sie die Regelsets \textit{RS-B0}, \textit{RS-B1} und \textit{RS-B2} implementieren. Im Gegensatz zu Pellenkoft wird im Regelset \textit{RS-B1} nicht die Regel Swap verwendet, sondern Left\-Associtativity (vgl. \ref{PellenkoftVsPyro}). Beide Regeln sind jedoch semantisch ähnlich genug und können mit Hilfe von Kommutativität in einander überführt werden. Ebenfalls ist eine genaue Unterscheidung der Reihenfolge der Anwendung, ähnlich wie bei Volcano nicht möglich. Alle Regeln laufen immer in einer vorgegeben Reihenfolge ab.






\subsection{Unvollständigkeit von \textit{RS-B2}}






Die Unvollständigkeit von \textit{RS-B2} wurde auf zwei Arten gezeigt. Auf der einen Seite wurde an Hand des Beispiel \ref{Incompleteness_RS} gezeigt. Im konkreten Beispiel sollte \textit{R1} und \textit{R4} getauscht werden, wobei das Regelset \textit{RS-B2} zum Einsatz kommen sollte. Anhand des gegebenen Join Graphen ist die Transformation zwar Teil des Suchraums, kann jedoch durch die Regeln nicht erreicht werden. Das Regelset ist daher unvollständig.

Ebenfalls wurden unterschiedliche Abfragen mit Hilfe des Optimierers optimiert und dahingehend geprüft, ob die selbe Menge an Plänen aus den unterschiedlichen Regelsets entstehen kann. Wie erwartet war die Menge der Pläne, die aus \textit{RS-B0} nur \textit{RS-B1} entstanden sind gleich, die Menge der \textit{RS-B2} Pläne stark reduziert. (vgl. \ref{tabelle)}.

Da nicht alle Pläne gefunden werden konnten, liegt es nahe, dass der optimale Plan möglicherweise nicht Teil des erforschten Suchraums ist. \cite{shanbhag2014optimizing} stellten fest, dass das Ergebnis mit \textit{RS-B2} erzeugt wurde bzgl. der berechneten Kosten um den Faktor 1.86 schlechter war, als die Pläne, die mit Hilfe eines vollständigen Regelsets erzeugt wurden. Wie genau die Zahl 1.86 zustande gekommen ist, ist nicht genauer begründet.

Es lässt sich also feststellen, dass mit Hilfe des Regelsets \textit{RS-B2} weniger Pläne im Suchraum gefunden werden konnten, als mit Hilfe des Regelsets \textit{RS-B1} bzw. \textit{RS-B2}. Außerdem wurde festgestellt, dass Pläne aus unvollständigen Regelsets nicht immer den optimalen Plan beinhalten.





\subsection{Unvollständigkeit von RS\-02}




\cite{shanbhag2014optimizing} stellt fest, dass RS-B0-CPS und RS-B1-CPS vollständig sind. Die Vollständigkeit von RS-B2-CPS wird jedoch in Frage gestellt und die Unvollständigkeit mit Hilfe eines Beispiels belegt. Als Beispiel dient eine Menge von Relationen, die mit Hilfe des Jointrees J (\ref{fig:Incompleteness_RS-B2-CPS}) miteinander gejoint sind. Der Initale Anfragebaum \textit{Q1} ist in \ref{fig:Incompleteness_RS-B2-CPS} dargestellt. Das gewünschte Ergebnis nach einer Transformation \textit{Q2}  findet sich in \ref{fig:Incompleteness_RS-B2-CPS}. 

Bei \textit{RS-B2-CPS} dürfen die Regeln \textit{R2}, \textit{R3}, \textit{R4} nur jeweils einmal auf einen Join-Operator angewendet werden. Keine der Regeln darf danach auf den neu generierten Operator angewendet werden. Die In \ref{fig:Q1} und \ref{fig:Q2} zeichnet sich dadurch aus, dass die Relationen \textit{R1} und \textit{R4} vertauscht sind.


Die Vollständigkeit wurde auch mit Hilfe des PyroJ Optimierers geprüft. Hierbei wurde an Hand von Star-Queries  die Unvollständigkeit belegt. Wie in Abb. \ref{CompletenessResults} zu sehen, wurden mit Hilfe von \textit{RS-B2} nur knapp die Hälfte aller äquivalenten Pläne gefunden. Ebenfalls wird betont, dass die geschätzten Kosten des besten Plans, der durch \textit{RS-B2} erzeugt wurde um den Faktor 1.86 im Vergleich zu \textit{RS-B1} niedrigere geschätzte Kosten hatte.







\subsection{Diskussion}
Die Implementierung der neuen Regel wurde in einem Java basierten regelbasierten Optimierer implementiert. Den Äquivalenzklassen wurden neue Felder hinzugefügt.

Die Generierung der Resultate findet ohne Pruning statt und Kosten für die Kostenberechnung werden nicht einbezogen. Ebenfalls bleiben Regeln bestehen, die für denn SELECT Pushdown verantwortlich sind, als Teil der Normalisierungsphase. 

Es wurden sowohl Star, Chain als auch Clique Queries getestet. Die Experimente fanden auf einem Intel i5 3.5 GHz mit 8 GB Ram statt. Es wurde festgestellt, dass die Geschwindigkeit der Optimierung verbessert werden konnte, dadurch, dass die Optimierung mehrfach durchgeführt wurde. Dieses Verhalten wurde auf Javas JIT Kompilierungsstrategie zurückgeführt, die erst den Code kompiliert, wenn er auch tatsächlich gebraucht wird. Um sicherzustellen, dass der Kompilierte Code nicht wieder während der Ausführung vergessen wird, mussten spezielle Flags für die JVM gesetzt werden. Das Ergebnis war, dass die Dauer zwar verglichen zu den schnellsten Tests ohne das Flag langsamer, aber dafür konstant blieb.

Ebenfalls wurde bei Regelmenge $RS_B1$ geprüft, ob der gesamte Search Space erreicht wurde, indem die Anzahl der Äquivalenten Knoten und Operatoren im LQDAG gezählt wurden. Diese Zahl wurde mit der Zahl der Knoten in RS-Graph verglichen. Da beide Zahlen gleich waren, wird davon ausgegangen, dass beide Regelsets das selbe Ergebnis erzeugen. Eine Prüfung, die dies belegt, fand aus technischen Gründen nicht statt.






