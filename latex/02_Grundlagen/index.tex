\newpage
\section{Grundlagen / Related Work}


In diesem Kapitel werden die Grundlagen erläutert und bereits bekannte Implementierungen erklärt.  Das Kapitel beginnt mit einer kurzen Übersicht über die Funktionsweise von Datenbanksystemen vor dem Hintergrund der Anfrageoptimierung. Im Zentrum stehen die Komponenten Query Optimizer und Query Execution Engine. Beide Teile sind integraler Bestandteil eines Datenbanksystems. Im weiteren wird der für die Anfragenoptimierung wichtige Begriff des Search Spaces eingegangen. Das Kapitel wird fortgesetzt mit einer Übersicht über bereits bestehende Datenbanksysteme und deren Optimierer. Es wird konkret auf IBMs Starburst-Projekt und das EXODUS Projekt \cite{graefe1987exodus}, \cite{carey1990exodus} mit seinen Nachfolgern Volcano \cite{graefe1990parallelizing}, \cite{graefe1990encapsulation}, \cite{graefe1993volcano}, \cite{graefe1994volcano} und Cascades \cite{graefe1995cascades} eingegangen. Abgerundet wird das Kapitel mit einer Erklärung der von Pellenkoft zusammengestellten Regelsets \cite{pellenkoft1997complexity}, \cite{pellenkoft1997duplicate}. Alle Komponenten zusammen bilden die Grundlage für die im nächsten Kapitel besprochene Implementierung der Planexpanders.




\subsection{Query Optimizer und Query Execution Engine}
Die Arbeit an Anfragenoptimierern wurde in den frühen 1970er Jahren begonnen und bis heute fortgesetzt. Dieses Kapitel beleuchtet die Geschichte der Anfragenoptimierung und basiert auf \cite{chaudhuri1998overview}.


Zu Beginn der Anfragenoptimierung steht die Anfrage. Eine Anfrage wird in einer Anfragesprache formuliert. Die Anfragesprache ist ein deklaratives, high-level Interface zum Zugriff auf Daten. In der Welt der relationalen Datenbanken hat sich SQL als Standardsprache durchgesetzt. Bei der Verarbeitung einer Anfrage durch ein DBMS sind zwei Komponenten entscheidend: (1) Der Query Optimizer  (QO) und (2) die Query Execution Engine (QEE).

Die QEE implementiert eine Reihe von physikalischen Operatoren. Die Aufgabe eines Operators ist es aus einem oder mehreren Eingabedatenströmen einen Ausgabedatenstrom zu erzeugen. Zu diesen physikalischen Operatoren gehören beispielsweise sort, index scan, sequencial scan, nested-loop join, und sort-merge. Es ist möglich, dass eine logische Operation als mehrere physiche Operationen ausgedrückt werden kann. Physische Operationen sind somit nicht 1:1 logischen Operationen zugeordnet. Physikalische Operatoren sind Codeblöcke die die Ausführung von Anfragen erlauben. Eine solche Anfrage lässt sich als Operatorenbaum darstellen und wird als Execution Plan bezeichnet. Die Ausführungsengine ist verantwortlich für die Ausführung eines Execution Plans und liefert als Resultat die Antwort auf eine Anfrage.

Die Aufgabe des QO ist es basierend auf einer zuvor geparsten Anfrage einen optimierten Plan zurückzuliefern, der als Input für den QEE dient. Als optimaler Plan gilt der, der am effektivsten durch den QEE ausgeführt werden kann. Die Aufgabe der Generierung des optimalen Plans ist nicht einfach, da eine Anfrage von vielen unterschiedlichen Plänen repräsentiert werden kann. Beispielsweise ist es möglich einen algebraischen Ausdruck in einen anderen logisch gleichen, jedoch in Form unterschiedlichen Ausdruck zu verwandeln. Ein Beispiel hierfür ist Kommutativität, auf die später genauer eingegangen wird:

$$ JOIN(A,B) = JOIN(B,A)$$

Ebenso ist es möglich, dass für die logische Repräsentation mehrere Execution Pläne zur Verfügung stehen. Beispielweise kann ein logischer JOIN von mehreren physikalischen JOIN Methoden repräsentiert werden.

Die Dauer der Ausführung eines Execution Plans kann von Plan zu Plan stark variieren. Daher ist es notwendig eine Auswahl zu treffen, welches der schnellste Plan für die Ausführung einer Anfrage ist. Um dieses Problem zu lösen, müssen drei Bereiche betrachtet werden:

\begin{itemize}
\item Der Suchraum (Search Space). Er bildet die Menge aller logisch äquivalenten Pläne.
\item Die Kostenschätzung (Cost Estimation). Ihre Aufgabe ist es die Kosten und schlussendlich die Dauer für die Ausführung eines Plans zu bestimmen.
\item Ein Enumeration Algorithm, der für die Suche innerhalb es Search Spaces verantwortlich ist.
\end{itemize}

Das Ziel eines jeden Optimierers ist es (1) einen Suchraum zu erzeugen, der den Plan enthält, der die niedrigsten Kosten verursacht, (2) eine akkurate Kostenschätzung abzugeben und (3) effizient nach dem günstigsten Plan zu suchen. Jeder dieser Ziele ist nicht trivial zu erreichen.



\subsection{Search Space}

Der Search Space bildet die Grundlage, innerhalb derer der Optimizer nach einem optimalen Plan suchen kann. Der Suchraum ist abhängig davon welche algebraischen Transformationen zur Anwendung kommen und welche physischen Operatoren durch den Optimierer unterstützt werden. Die Transformation von Plänen sorgt dabei nicht per se für einen optimierten und optimaleren Baum, sondern bildet nur Alternativen. Aus diesen wird mittels Kostenfunktionen und Enummeratoren der optimale Plan ausgewählt.

Einige Optimierer nutzen während der Verarbeitung einer Anfrage unterschiedliche Repräsentationsformen. Als initiale Repräsentationsform steht meist eine geparste Anfrage zur Verfügung. Das Resultat ist i.d.R. ein Operatorenbaum. Während der Verarbeitung durch den Optimierer kann zwischendurch auch ein logischer Operatorenbaum (auch Query Tree) genutzt werden.

Einige Systeme nutzen auch kalkülorientierte Repräsentationsformen zur Analyse der Struktur einer Anfrage. Solche Anfragen lassen sich in der Form von Query Graphen darstellen, deren Knoten Relationen repräsentieren und deren Kanten mit Join Prädikaten gekennzeichnet sind. So einfach diese Repräsentationsform ist, so schwierig ist diese in der Umsetzung. Diese Prädikate Graphen repräsentieren nur einen Teil der möglichen Operationen. Ein solcher Graph kann zwar natürliche Joins darstellen, die Repräsentation von UNION ist nicht möglich. Ebenfalls lassen sich verschachtelte Anfragen nicht darstellen. Diesem Nachteil tritt der GQM des Starburst Projektes entgegen. Ein GQM ist eine erweiterte Form der Query Graphen. Der GQM erlaubt es mit Hilfe von Building blocks einfache SQL Statements zu repräsetieren und weitere nested Query als Subgraphen darzustellen. Im Gegensatz dazu nutzt EXODUS und seine Nachfolger von vornherein einen uniformen Querybaum und Operatorenbaun für alle Phasen der Optimierung.



\subsection{Trees}

[HIER MEHR TEXT]
\cite{ioannidis1991left}
\begin{itemize}
\item Left-Deep / Right Deep
\item Zick-zack Tree
\item Bushytree
\end{itemize}

\section{Das Starburst Project}

Das Starburst Projekt \cite{lohman1988Starbust}, \cite{haas1989extensible}, vorangetrieben und entwickelt von IBM,  startet unter der Prämisse, dass bestehende DBMSe nicht in der Lage sind die wachsenden Anforderungen von verschiedenen, neuartigen Applikationen vollumfänglich zu entsprechen. Zur Erfüllung der individuellen Ansprüche wurde das Starburst Projekt begonnen. Sein Zeil ist des dem \ac{DBI} die Möglichkeit zu geben eine die relationale Datenbank zu erweitern und so die Bedürfnisse von spezifischen Anwendungen zu erüllen. Beispielsweise ist es möglich neue Zugriffs- und Speichermethoden zu implementieren oder neue Join Methoden zu erstellen. Um diese Features zu ermöglichen stellt Starburst eine Anfragesprache, einen Regelbasierten Optimierer, Query Rewriter und ein Ausführungssystem basierend auf relationaler Algebra zur Verfügung. Dieses Kapitel befasst sich zuerst mit der Aufteilung zwischen Ausführung und Anfrage, wie sie bereits zu Beginn von Kapitel 2 besprochen wurde. Es folgt ein Überblick über die eingesetzte Regelmaschine und eine Erläuterung des Starbust Query Optimizers.

\subsection{Ausführung einer Anfrage}

Bei der Ausführung einer Anfrage mit Hilfe von Starburst wird grob zwischen zwei Phasen unterschieden: Übersetzungs- (Compile-) und Ausführungszeit (Run-Time). Während der Compile-Time wird aus der Anfrage ein Plan generiert, der zur Run-Time durch das Execution System ausgeführt wird. Der Query Optimizer findet seine Anwendung zur Compile-Time; die Query Execution Engine kommt zur Run-Time zum Einsatz.

\begin{figure}[h]
  \centering
  \includegraphics[width=\textwidth]{02_Grundlagen/Starburst.png}
  \caption{Starburst}
\end{figure}

Zur Comile-Time wird eine gegebene Anfrage auf semantische Korrektheit geprüft und in ein Query Graph Model (QGM) übersetzt. Basierend auf diesem QGM wird die Optimierungen der Anfrage durch zuerst einen Query Rewriter, einen Plan Optimierer und einen Plan Refiner durchgeführt.

Der Query Rewriter hat zwei konkrete Aufgaben: Auf der einen Seite soll die Anfrage in eine möglichst deklarative Form umgewandelt werden, hierbei werden insbesondere Anfragen entschachtelt. Auf der anderen Seite sollen weithin in der Forschung akzeptierte Heuristiken, beispielweise der Push-Down von Argumenten, angewandt werden.
Bei der Planoptimierung durchläuft der Plan vom QGM hin zu einem QEM drei Stationen. Die drei wesentlichen Aspekte sind der Plan Generator, die Berechnung der Plankosten und die Suchstrategie. 

\subsection{Regel-Maschine}

Zur Ausführung von Optimierungen wurde für die Starburst Datenbank eine eigene Regel-Maschine entwickelt \cite{lohman1988Starbust}. Diese Regelmaschine ist für die Anführung von Regelsets verantwortlich, die bei der Transformation der QGMs zur Anwendung kommen und durch den DBI  erweitert werden können. Die Regelmaschine basiert auf fünf Prinzipien:

\begin{enumerate}
\item Regel der arbiträren Komplexität: Eine Regel ist bei Starburst in zwei Teile aufgeteilt: Eine Koordinierungs- und eine Ausführungsfunktion. Bei dem Aufruf einer Regel muss zuerst geprüft werden, in wie weit die Regel angewendet werden darf. Ist sie anwendbar, wird die Ausführungsfunktion exekutiert.

\item Bei der Nutzung eines GQMs kann entweder eine Tiefen- als auch eine Breitensuche zur Anwendung kommen. Diese Art der Suche ist weder an die Regel noch an das Regel-Set geknüpft, sondern mit dem QGM selbst verbunden.


\item Die Regeln werden bei Starburst in Regelsets unterteilt. Ein Regel-Set besteht aus mehreren Regeln, die entweder sequenziell, nach einer vorab vergebenen Priorität oder einem statistischen Verfahren aufgerufen werden. Jede Regel und jedes Regel-Set kann andere Regeln und Regel-Sets als Teil einer Subroutine ausführen. _Neben der Funktion der Zusammenfassung der Regeln ist es Aufgabe der Regel-Sets._

\item Um die Ausführung von Transformationen zu limitieren, insbesondere bei der Ausführung des Rewriters, kann ein Budget vorgeschrieben werden. Sobald dieses Zeitbudget aufgebraucht ist, wird die Ausführung von Transformationsregeln abgebrochen. Es ist für diese Funktion unbedingt notwendig, dass Regeln immer vollständige QGMs zurückliefern, da sonst ein unvollständiger QGM als Resultat des Optimierers entstehen kann.

\item Schlussendlich ist es dem Nutzer der Datenbank möglich zu jedem beliebigen Zeitpunkt Regeln ausser Kraft zu setzen. Dies geschieht nur für den Nutzer lokal und andere Nutzer der Datenbank sind nicht betroffen. Spezielle Anwendungsfälle können so eine passgenau optimiert werden.
\end{enumerate}

\subsection{Plan Optimierer}

Der Plan Optimierer generiert basierend auf einem QGM mehrere alternative Query Evaluation Plans (QEPs). Für jeden dieser QEPs werden die Kosten geschätzt und der günstigste für die Weiterverarbeitung ausgewählt. Um die Erweiterbarkeit des Optimierers zu gewährleisten wurden die drei Komponenten (Plangenerierung, Kostenschätzung und Suchstrategie) von einander soweit getrennt, dass sie einzeln erweitert und verändert werden können. 

\subsection{Starburst Plan Generator}
$$HIER ÜBERARBEITEN$$
Der Plangenerator des Starburst Projekts, der für die Generierung von Planalternativen verantwortlich ist, bedient sich eines "Build Block"-Ansatzes[Lohm88] und einem Regel-Set, das auf einer eigenen Grammatik basiert.  

Als Build Block werden low-level-Datenbankoperationen (wie Access, Join und Sort) zu high-level Operationen kombiniert. Diese können wiederverwendet werden. Dank des Konzepts der Build Blocks ist es möglich die Erstellung und Verarbeitung in zwei wesendlichen Aspekten zu vereinfachen:


\begin{enumerate}


\item Die Regeln sind leichter von einem DBI zu lesen und zu verstehen, da die Fülle an Information mit Hilfe von Build Blocks aggregiert wurde.

\item Das Ausführen von Regeln wird effizienter. Da nicht mehr ganze Graphen nach Pattern durchsucht werden müssen, sondern direkt über Build Blocks erkannt werden, erleichtert sich die Ausführung. Ebenfalls ist es dank der Nutzung von Macro-Expandern möglich, die Geschwindigkeit zu verbessern.


\end{enumerate}


$$TEXT$$
Jede Regel erlabt es, dass aus ihr sowohl eine als auch mehrere Graphen entstehen können. In diesem Falle spricht man von Sets von Alternativen Plänen (SAP). 


\begin{itemize}
\item LOLEPOP
\item STARs
\end{itemize}

 

\subsection{Exodus, Volcano, Cascades}

Das Volcano Projekt mit seinem Vorgänger \ac{EXODUS} und dem Nachfolger Cascades bildet die Grundlage für den Microsoft SQL Server und dessen Anfrageoptimierer. Bei der Implementierung des Volcano Optimierers ist insbesondere die Implementierung des Query Optimierers von Interesse und wird daher ausführlich behandelt.

\subsubsection{Exodus}


Bereits in den 1970er Jahren begann Graefe mit der Implementierung eines DBMS Frameworks unter dem Titel EXODUS \cite{carey1990exodus}.  Das Projekt, das die Grundlage für Volcano legen sollte, hatte sich zum Ziel gesetzt, einen erweiterbaren, applikationsspezifischen und hochperformanten Baukasten zusammenzustellen, mit dessen Hilfe neue Datenbanksysteme generiert werden konnten. 

Im Gegensatz zu konventionellen DBMS wie System R oder Starburst handelt es sich bei EXODUS nicht um ein funktionsfähiges und sofort einsatzfähiges DBMS, sondern um einen Baukasten, auf dessen Basis ein neues System durch einen \ac{DBI} erstellt werden kann. Im Gegensatz zu anwendungsübergreifend designten DBMS wie Postgres bietet EXODUS den Vorteil, dass eine Datenbank speziell an die Bedürfnisse eines Anwendungsfalles angepasst ist und so die Anforderungen einer Anwendung passgenau erfüllt. Um dennoch das Ziel der Performance nicht aus den Augen zu verlieren, werden viele Komponenten nicht immer wieder auf einer grünen Wiese entwickelt, sondern auf dem Fundament des EXODUS Baukastens aufgebaut.

\begin{figure}[ht]
  \centering
  \includegraphics[width=\textwidth]{02_Related_Work/ExodusDatabaseSystemStructure.png}
  \caption{Exodus Database System Structure}
  \label{ExodusDatabaseStructure}
\end{figure}\todo{Eigene Grafik bauen}

Der Baukasten von EXODUS besteht aus drei Arten von Bausteinen: Bausteine die fix vorgegeben und nicht verändert werden sollen, Bausteinen, die speziell entwickelt werden müssen und Teilen, die generiert werden. Der Werkzeugkasten umfasst dabei nicht nur die Bausteine, sondern auch die Werkzeuge zur Bearbeitung und Generierung. Zu den Werkzeugen gehören ein Tool zur Erstellung eines Front-Ends für die Anfragesprache, ein Query Optimizer Generator und die Programmiersprache E (zusammen mit einem passenden Compiler). Mit Hilfe des Tools zur Erstellung eines Front-Ends für Anfragesprachen kann die Parser Komponente generiert werden. Der Query Optimizer wird als Resultat des Query Optimizer Generators erzeugt. (vgl. \ref{ExodusDatabaseStructure})

Neben den generierten Komponenten gibt es den E Compiler, der E Code in Objekt-Code übersetzt. Er kommt zum Einsatz, um die durch den Query Optimizer optimierte Anfrage in eine kompilierte Anfrage umzusetzen. Diese Komponente ist ähnlich wie der Storage Manager, der für die Verwaltung von Daten in der Datenbank genutzt wird, unveränderlich. 

Zwischen der kompilierten Anfrage und dem Storage Manager kommen zwei Komponenten zum Einsatz, die von einem DBI geschrieben werden müssen: Die Operator Methoden und Access Methoden. Diese beiden Komponenten dienen dazu, die Anfrage in Code zu übersetzen, der durch den  Storage Manager ausgeführt wird.

Bei der Implementierung eines Optimierers kommen grundsätzlich zwei mögliche Ansätze in Frage: (1) interpretierte und (2) kompilierte Programmiersprachen. Bei EXODUS wurde zuerst die Implementierung mittels sog.\"AI\" Sprachen versucht. In einem Prototypen wurde mit Hilfe von Prolog ein Optimierer entwickelt. Für Prolog hat man sich entschieden, da diese Sprache Pattern Matching und eine Search Engine bereitstellt. (Auch unification konnte zum Einsatz gebracht werden, um elegant Query Trees zu erstellen.) Der Hauptvorteil eines interpretierten Ansatzes war aus Sicht von EXODUS die Möglichkeit neue Regeln zur Laufzeit des Programms hinzuzufügen. Trotz dieser Vorteile wurde der Ansatz als zu langsam verworfen. Auch der Vorteil Regeln während der Laufzeit hinzuzufügen, wird in der Literatur \todo{QUELLE} als wenig nützlich bewertet. Statt dieses Ansatzes wurde in der Folge auf die Erstellung eines Generators, der in C geschrieben wurde, gesetzt. Der Generator erstellt basierend auf Regeln einen Optimierer in C, der wiederum kompiliert werden kann. Zwar war die Entwicklung des C Generators aufwändiger als die Implementierung in Prolog, jedoch konnte auf applikationsspezifische Notwendigkeiten, wie die Implementierung von speziellen Suchverfahren, punktgenau eingegangen werden.


Die Generierung des Optimierers geschieht auf Basis eines Description-Files. Dieses File setzt sich aus drei Komponenten zusammen: Operatoren, Methoden und Regeln.


Die Operatoren bezeichnen die logische Algebra. Die Methoden werden in der physischen Algebra verwendet. Ebenfalls sind Transformationen Teil des Description-Files. Es wird zwischen zwei Arten an Regeln unterschieden: Transformationsregeln und Implementierungsregeln. Implementierungsregeln kommen bei der Übersetzung von logischer in physische Algebra zum Einsatz. Transformationsregeln sind für die Erzeugung von alternativen Plänen verantwortlich.

Allen Regeln ist gemein, dass sie wie bereits beschrieben aus zwei Teilen bestehen. Einer Methode, die in C geschrieben ist, die prüft, ob eine Regel angewendet werden kann und einer Transformation, die auf einer algebraischen Äquivalenz basiert. Der linke Teil der Äquivalenz stellt dabei die Eingabe der Transformation dar, der rechte Teil der Äquivalenz den Output der durch Transformation entstanden ist. Neben dieser deklarativen Implementierung von Regeln ist es auch möglich komplexere Regeln mit Hilfe von C Routinen zu entwickeln.

Diese Regeln werden gemeinsam mit Hilfe des Optimizer Generators in C Code umgewandelt, der dann wie jeder andere Query-Optimizer verwendet werden kann. 

Bei der Anwendung der Regeln wird zuerst geprüft, welche Regeln anwendbar sind. Diese Regeln werden in der Liste OPEN abgelegt. Durch einen Auswahlmechanismus wird festgelegt, welche Regel zum Einsatz kommt. Nachdem eine Regel angewendet wurde, wird sie aus der OPEN Liste entfernt. Regelanwendungen, die durch diese Transformation möglich wurden, werden der OPEN Liste hinzugefügt. Der Auswahlmechanismus sieht vor, dass die potenziell gebildeteten Pläne auf Grund ihrer Kosten bewertet werden und die Regel ausgeführt wird, die zum günstigsten Plan führt. Die Suchstrategie wird als Hill climbing bezeichnet.

Einer der Nachteile dieses Ansatzes ist, dass Pläne nicht bzgl. ihrer Güte auf einem globalen Level bewertet werden können. Es ist nicht möglich festzustellen, ob es sich um den absolut günstigsten Plan handelt, da weitere Transformationen noch angewendet werden können. Auch Kostenreduktionen, die erst durch die Kombination mehrerer Regeln entstehen, werden nicht berücksichtigt. Auch Graefe selbst sieht einige weitere Nachteile \cite{graefe1993volcano}:

\begin{itemize}
\item nicht-triviale Kostenmodelle
\item keine Eigenschaften
\item keine Heuristiken
\item keine Transformation von Subscripten von algebraischen Operatoren nach algebraische Operatoren
\end{itemize}


\subsubsection{Volcano Optimizer}

Volcano ist der verbesserte Nachfolger von EXODUS. Zuerst war Volcano nur ein erweiterbares, paralleles System zur Anfragenausführung. Später wurde ein neuer Anfrageoptimierer-Generator hinzugefügt. Im Gegensatz zu der bisher verwendeteten Programmiersprache E in der der Optimierer erzeugt wurde, setzt das Projekt auf C Code. 


Bei der Generierung eines Plans auf Grund einer Anfrage kommt bei Volcano ein Optimierer zur Anwendung, der speziell für den Anwendungsfall generiert wurde. Der Optimierer ist das Resultat einer Model Spezifikation, die von einem \ac{DBI} erstellt und durch einen Optimierer Generator in Source Code umgewandelt wird. Dieser Code wird mit Hilfe eines Compilers in das Endprodukt, den Optimierer, umgewandelt.


Der Volcano Optimizer wurde von Grund auf neu entwickelt und begegnet den von Graefe identifizierten Nachteilen des EXODUS Optimizer Generators. Ebenfalls wird die Arbeit mit der Sprache E eingestellt und vollumfänglich auf C gesetzt. Die Verbesserungen werden in fünf fundamentalen Designentscheidungen\cite{graefe1993volcano} sichbar, die eine effiziente Suche des Optimierers erlauben:


\begin{enumerate}

\item Die erste Grundlage des Volcano Optimierers ist die Anwendung von algebraischen Techniken wie algebraische Operatoren und Äquivalenzklassen. Volcano unterscheidet dabei zwischen logischer und physischer Algebra. Die Umwandlung von logischer Algebra (der Anfrage) in die physische Algebra (einem Query Evaluation Plans) geschieht durch die Transformation der logischen Algebra mit Hilfe von kostenbasierten Mappings von logischen Algorithmen. \todo{Unklar}


\item Das zweite Prinzip sieht vor, dass die Information über algebraische Gesetze, die zur Transformation von Algebraischen Ausdrücken genutzt werden als Regeln und Pattern modular erfasst sind. Durch dieses Prinzip sind die einzelnen Regeln klar und transparent von einander getrennt und können zu Regelsets zusammengestellt werden. Einige dieser Regelsets werden im folgenden Kapitel behandelt.


\item Das dritte Prinzip betrifft die Eingabe des Optimierers. Im Gegensatz zu anderen Optimierern (namentlich Starbust) setzt der Volcano Optimizer auf algebraische Äquivalenzen als Eingabe Parameter. Andere Systeme nutzen hier mehrere Stufen an Umwandlung, um zwischen Query und Optimizer Eingabe zu vermitteln. 

\item Kompilierung über Interpretierung von Regeln ist das vierte Prinzip, das zur Anwendung kommt. Da es sich bei der Generierung von äquivalenten Plänen um ein CPU intensives Geschäft handelt, wurde entschieden, dass die Regeln zur Transformation der Pläne kompiliert und nicht interpretiert werden. Zwar verliert der Optimierer daduch die Möglichkeit ad hoc neue Regeln in den Optimierer aufzunehmen. Jedoch wird diese Möglichkeit in der Praxis nicht benötigt. 

\item Das letzte Prinzip ist, dass der Volcano Optimierer auf dynamisches Programmieren bei der Generierung von Programmen setzt. \todo{Unklar}
\end{enumerate}


Wie bereits zuvor wird der Ausdruck in einen Operatorbaum umgewandelt. Transformationsregeln und Implementierungsregeln kommen bei der Optimierung zum Einsatz. Eine Trennung zwischen Optimierung auf logischer und auf physischer Ebene findet nicht statt. Es wird zuerst ein logischer Ast erzeugt, für den dann unterschiedliche physische Pläne generiert werden. Diese Pläne werden auf der Basis ihrer Kosten verglichen, nur der günstigste Plan bleibt bestehen und wird gespeichert. Ist bereits ein Plan vorhanden, so wird nur der günstigste Plan im Speicher behalten. Bei der Generierung kommt zudem eine Memo-Struktur zum Einsatz. In einer Hashtabelle wird geprüft ob eine Transformation bereits bekannt ist. Wenn das der Fall ist, wird diese Transformation verwendet und somit doppelter Berechnungsaufwand gespart und die bekannte Transformation stattdessen referenziert.

Regeln und insbesondere Transformationsregeln, die bei der Transformation im Bereich JOIN-Reihenfolgeänderung verwendet werden, werden in \ref{sec:pellenkoftRulesets} beschrieben.



\subsection{Pellenkoft Rulsets}

Bei der Erforschung eines Search Spaces kommen in Transformations-basierten Query Optimierern Transformationsregeln zum Einsatz. Pellenkoft et al. \cite{pellenkoft1997duplicate} \cite{manegold2000multi} \cite{pellenkoft1997complexity} stellt drei Regelsets zur Verfügung, die bei der Erzeugung eines Search Spaces zum Einsatz kommen können.

Zu Beginn besteht ein Search Space aus einem Plan. Dieser Plan stellt die Eingabe dar. Auf Ihn werden Transformationsregeln angewendet. Neue Pläne entstehen. Sollten diese Pläne noch nicht vorhanden sein, werden sie dem Suchraum hinzugefügt. Pläne auf die bereits alle anwendbaren Transformationsregeln auch angewendet wurden werden als besuchte Pläne bezeichnet. Sobald alle Pläne besucht wurden und somit auf alle Pläne Transformationsregeln angewendet wurden, ist der Suchraum vollständig erforscht und keine weiteren Pläne können gefunden werden. 

Mehrere Transformationsregeln werden gemeinsam als Regelsets bezeichnet. Pellenkoft et al. unterscheidet zwischen Regelsets, die Pläne mehrfach generieren können und Regelsets, die duplikatfrei sind. Beispielsweise kann durch die Anwendung der Regel Kommutativität auf einen Plan und erneute Anwendung auf dessen Resultat wiederum der ursprüngliche Plan generiert werden. Im Folgenden werden zwei Regelsets vorgestellt, die Duplikate bilden und ein Regelset, das duplikatsfrei ist.


\subsubsection{Regelset mit Duplikaten}

Eines der Regelsets, das zur Erzeugung eines Bushy Space genutzt werden kann, ist RS-B0. Es besteht aus drei Regeln:

\begin{itemize}
\item Kommutativität: $$ A \Join B \to B \Join A$$
\item Rechte Assoziativität: $$(A \Join B) \Join C \to A \Join (B \Join C) $$
\item Linke Assoziativität: $$A \Join (B \Join C) \to (A \Join B) \Join C$$
\end{itemize}

Das Regelset ist redundant, da mit Hilfe von Kommutativität und rechter Assoziativität. linke Assoziatvität (und vis-à-vis) erzeugt werden kann. Das daraus abgeleitete Regelset RS-B1 besteht daher aus folgenden Regeln:

\begin{itemize}
\item Swap $$ (A \Join B) \Join C \to (A \Join C) \Join B $$
\item Bottom Commutativitity $$ B_1 \Join B_2 \to B_2 \Join B_1$$
\end{itemize}

Durch die Anwendung der Regeln aus RS-B0 und RS-B1 können Pläne doppelt erzeugt werden. Am einfachsten ist dies an Hand von Kommutativität zu zeigen. Wird auf den Plan a JOIN b Kommutativität angewendet, entsteht b JOIN a, dann entsteht durch  die erneute Anwendung von Kommutativität auf den neuen Plan B JOIN A wieder der ursprüngliche Plan.

Ebenfalls können sich bei komplexeren Plänen Teilpläne gleichen. Beispielsweise enthält der Plan (A JOIN B) JOIN C den gleichen Subplan wie C JOIN (A JOIN B). Um solche Duplikate zu verhindern, wird von Pellenkoft das Prinzip der Äquivalenzklasse  angewendet.



\subsubsection{Duplikatfreie Regelsets}
Durch die Anwendung on RS-B0 bzw. RS-B1 ist es möglich, dass Varianten des Plans erneut erzeugt werden. Dieser Gefahr trägt das Regelset RS-B2 Rechnung. Es sieht vor, dass eine Regel nur genau einmal ausgeführt und andere Regeln nur einmal pro Operator ausgeführt werden dürfen. Dieses Regelset besteht aus:


\begin{itemize}
\item Kommutativität: $$ A \Join B \to B \Join A$$
\item Rechte Assoziativität: $$(A \Join B) \Join C \to A \Join (B \Join C) $$
\item Linke Assoziativität: $$A \Join (B \Join C) \to (A \Join B) \Join C$$

\item Exchange $$(A \Join B) \Join (C \Join D) \to (A \Join D) \Join (C \Join B) $$
\end{itemize}



\subsubsection{Kreuzproduktfreie Regelsets und Vollständigkeit}

Die bisherigen Regelsets können zu Kreuzprodukten führen. Ebenfalls ist nicht geklärt, ob die Regelsets vollständig sind und alle möglichen kreuzproduktfreien Bäume erzeugen. Im Folgenden wird zuerst der Begriff der Kreuzproduktfreiheit eingeführt und basierend auf diesem Begriff die Vollständigkeit erläutert.

Ein Kreuprodukt kann bei den vorliegenden Plänen dann entstehen, wenn durch die Anwendung einer Regel ein Join zwischen zwei Relationen gebildet wird, die zuvor keine Kante auf einem gegeben Join Tree hatten.

Eine Technik um kreuzproduktfreiheit bei den bisherigen Regelsets herzustellen, ist die Unterdrückung von Kreuprodukten die auch als \ac{CPS} bezeichnet wird. Der Ansatz funktioniert so, dass eine Regel, falls sie ein Kreuzprodukt erzeugt zwar als ausgeführt markiert, jedoch nicht der Baum in den Search Space aufgenommen wird. Somit werden Kreuzprodukte zwar erzeugt, aber nicht in den Suchrraum aufgenommen. Regelsets, die diese Art von Kreuzproduktunterdrückung anwenden, erhalten das Suffix CPS. Somit entstehen aus den Regelsets RS-B0, RS-B1 und RS-B2 die Regelsets RS-B0-CPS, RS-B1-CPS und RS-B2-CPS.


Eine andere wichtige Eigenschaft von Regelsets ist Vollständigkeit. Sie setzt voraus, dass Kreuzproduktfreiheit gegeben ist. Vollständigkeit liegt dann vor, wenn alle kreuzproduktfreien Pläne in einem Suchrraum erzeugt wurden.


\subsubsection{Unvollständigkeit von RS-02}


\begin{figure}[h]
  \centering
  \includegraphics[width=\textwidth]{03_Related_Work/Incompleteness_RS-B2-CPS.png}
  \caption{Incompletness of RS-B2-CPS}
  \label{Incompleteness_RS-B2-CPS}
\end{figure}


\cite{shanbhag2014optimizing} stellt fest, dass RS-B0-CPS und RS-B1-CPS vollständig sind. Die Vollständigkeit von RS-B2-CPS wird jedoch in Frage gestellt und die Unvollständigkeit mit Hilfe eines Beispiels belegt. Als Beispiel dient eine Menge von Relationen, die mit Hilfe des Jointrees J (\ref{fig:Incompleteness_RS-B2-CPS}) miteinander gejoint sind. Der Initale Anfragebaum $Q1$ ist in \ref{fig:Incompleteness_RS-B2-CPS} dargestellt. Das gewünschte Ergebnis nach einer Transformation $Q2$  findet sich in \ref{fig:Incompleteness_RS-B2-CPS}. 

Bei RS-B2-CPS dürfen die Regeln R2, R3, R4 nur jeweils einmal auf einen Join-Operator angewendet werden. Keine der Regeln darf danach auf den neu generierten Operator angewendet werden. Die In \ref{fig:Q1} und \ref{fig:Q2} zeichnet sich dadurch aus, dass die Relationen $R1$ und $R4$ vertauscht sind.







\subsubsection{Vorschlag von RS-Graph}

\cite{shanbhag2014optimizing}
