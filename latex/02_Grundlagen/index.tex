\chapter{Related Work}


In diesem Kapitel werden die Grundlagen für die folgenden Kapitel gelegt und Hintergründe erläutert. Im ersten Teil wird zuerst ein Überblick über die Architektur eines Datenbanksystems vor dem Hintergrund der Anfragenoptimierung gegeben. Der für die Optimierung wichtige Begriff des Search Spaces erläutert. In ihm können mit Hilfe von Enumeratoren alternative, äquivalente Pläne gefunden werden und aus diesen Plänen mit Hilfe einer Kostenschätzung der optimale Plan gefunden werden. Auch die Techniken der Enumeration und Kostenschätzung werden in ihrer Funktion beschrieben.

Neben diesen Grundbegriffen werden auch konkrete Implementierungen von DBMSen und deren Anfragenoptimierern besprochen. Neben System R, das die Grundlage für alle weiteren Anfragenoptimierer bildet, wird auch IBMs Starburst Projekt, die Projekte rund um Volcano und Oracles DBMS besprochen und deren unterschiede aufgezeigt.

Im nächsten Schritt wird auf die konkrete Implementierung der Suche nach Alternativen Plänen am Beispiel von Volcano und dessen regelbasierten Anfragenoptimierer. Die Regelsets von \cite{pellenkoft1997complexity}, \cite{pellenkoft1997duplicate} werden abgegbildet.

Nachdem der Aufbau von DBMSen, Anfrageoptimierern und konkrete Beispiele bekannt sind, wird auf die Arbeit von \cite{indien} eingegangen. Die Unvollständigkeit eines Regelsets wird beleuchtet, der Versuchsaufbau besprochen und das neue Regelset erläutert.

All diese Bestandteile bilden die Grundlage und den weiteren Kontext für die Implementierung des Regelbasierten Join Enumerators, der im nächsten Kapitel vorgestellt wird.






\section{Grundlagen und Grundbegriffe}

In diesem Teil werden die theoretischen Grundlagen gelegt. Die Architektur eines DBMS ist zentral für die Einordnung der einzelnen Komponenten. Der Search Space ist der Raum in dem der optimale Plan zur Ausführung in der Datenbank gefunden wird. Wie dieser Plan gefunden wird, legt der Enumerator fest und desssen Bewertung wid durch eine Kostenschätzung vorgenommen.


\section{Komponenten Optimierer Prototypen}




\begin{figure}[ht]
  \centering
  \includegraphics[width=\textwidth]{04_Implementierung/00_media/Organization.pdf}
  \caption{Projektorganisation}
  \label{ProjectOrga}
\end{figure}


Das Projekt ist entlang der unterschiedlichen Aufgaben modular organisiert. Unter dem Dach der Anwendung finden sich vier unterschiedliche Sparten, die gemeinsam für das Ausführen des Programms verantwortlich sind: Planknoten und Äuqivalenzklassen, Regeln und Regelmengen, Ineratoren und Services. Die genaue Zuordnung zu den jeweiligen Bereichen ist in Abb. \ref{ProjectOrga} nachvollziehbar.




Bei der Ausführung eines Tests  müssen alle Bereiche zusammenarbeiten. Der Service-Bereich und insbesondere das Konfigurationsmodul sind für die konkrete Zusammenstellung der für den Test notwendigen Komponenten verantwortlich. Er entscheidet, welche Regeln zum Einsatz kommen, welcher Enumerator gewählt wird, welche konkreten Pläne getestet und mit Hilfe welcher Adapter die Daten ein-  bzw. ausgegeben werden. Um zu verstehen, welche Möglichkeiten das implementierte System bietet, sind die einzelnen Komponenten im Folgenden im Detail beschrieben.




\subsection{Adapter}

Die Implementierung bietet mehrere Adapter, die zur Umwandlung von externen Formaten in eine interne Repräsentationsform, oder von einer internen Repräsentationsform in ein externes Format genutzt werden können. Sie bieten die Möglichkeit andere Systeme an das bestehende System anzudocken und sorgen so für den notwendigen Anschluss und die Erweiterbarkeit durch Dritte.

Insgesamt werden drei Adapter mitgeliefert. (1) JSON\-Adapter, (2) String\-Adapter, (3) DOT\-Adapter.

Der Json-Adapter erlaubt es Daten im JSON Format zu importieren und wird beim Einlesen des initalen Plans genutzt. Er wandelt auf der Basis des Konfigurationsfiles JSON in Planknoten und Äquivalenzklassen um, die dann weiterverarbeitet werden können. Ebenfalls ist es möglich, Pläne in JSON auszugeben. Zu diesem Zweck implementiert der Parser auch eine \texttt{dump} Methode.

Neben dem JSON\-Adapter wird auch ein String Adapter verwendet. Er ist für die Ausgabe von Plänen als String verantwortlich. Im Gegensatz zu einem JSON\-Adapter ist die Eingabe von Plänen mit Hilfe dieses Moduls nicht möglich. Auch der DOT\-Adapter erlaubt nur die Ausgabe von Plänen im DOT\-Format, die  zur Generierung von graphischen Ausgaben verwendet werden können.

Eine weitere Aufgabe eines solchen Adapters kann auch die Übersetzung von Relationsnamen in Bitvektoren sein. Da das vorliegende System, wie in \ref{sec:Bitvector} beschrieben, Relationen als Bitvektoren abbildet, mag es nützlich sein, Relationsnamen in Bitvektoren zu übersetzen. Für diese Übersetzung sind auch Adapter vorgesehen, die zusätzlich implementiert werden können.












\subsubsection{Erweiterbarkeit von Regelsets}





\subsubsection{Ausführung von Regeln}

Das eigentliche Ausführen der Regeln wird durch einen XY durchgeführt. Im konkreten Fall kommt hier der Algorithmus ExhaustiveTransformation zum Einsatz. Der Algorithmus startet mit einer Äquivalenzklasse. Innerhalb dieser Äquivalenzklasse werden die Regeln, die durch ein Regelset vorgegeben sind auf einem PlanKnoten ausgeführt. Die Ausführung geschieht hierbei zuerst auf den Oberen Ebenen und setzt sich dann auf den Kindern eines Knoten fort. Somit können bei der Transformation eines gegebenen Baums alle Regeln auf andere Bäume angewendet werden.

Wichtig ist hierbei zu bemerken, dass dieser Algorithmus immer zuerst prüft, ob eine Regel auch tatsächlich für die Anwendung geeignet ist und dann erst der Algorithmus ausgeführt wird. Neben der eigentlichen Eignung wird auch geprüft, ob eine Äquivalenzklasse bereits vollständig expandiert wurde. Falls dies der Fall ist, wird von einer weiteren Anwendung von Regeln abgesehen. Diese Funktion kann insbesondere Vorteile bei der Implementierung von neuen Regelsets bieten. Nutzt ein gegebenes Regelset die Möglichkeit nicht nur einen neuen Planknoten zu generieren, sondern gleich mehrere Planknoten zu erstellen und auch in diesem Zusammenhang bereits mehrere Kinder-Knoten zu erstellen, kann die Reihenfolge der Expansion von Äuqivalenzklassen geändert werden. Die einzelnen Äquivalenzklassen, die bereits durch eine Regel expandiert wurden, werden als solche markiert und die bisher vorhandenen Regeln werden nicht mehr ausgeführt.



















\subsection{Kostenschätzung und statistische Informationen}



Die Kostenschätzung zur Suche des optimalen Plans geschieht bei der vorliegenden Implementierung in einem eigenen Kostenschätzungsmodul.Teil des Kostenmoduls ist ein Katalog innerhalb dessen Informationen über Selektivität von JOINs und Kardinalitäten von Relationen gespeichert sind. Diese Informationen werden aus dem Katalog mit Hilfe des Kostenschätzers ausgelesen und die Kosten für einen Planknoten berechnet.

Die Berechnung der Kardinalität findet bottom up statt. Für jede Äquivalenzklasse wird die optimale Kardinalität berechnet und gespeichert. Der Planknoten mit dessen Hilfe diese optimale Kardinalität erreicht wird, wird als bester Planknoten markiert. Folgt man dem Pfad der besten Planknoten ergibt sich der kostenoptimale Baum an Planknoten. Die Berechnung sieht vor, dass immer die Selektivität eines Operators mit dem Produkt der Kardinalität der Eingabeparameter multipliziert wird. Beispielsweise wird auf der untersten Ebene - der Ebene der Scans - für eine gegebene Relation die Kardinalität mit der Selektivität von 1 multipliziert, da die gesamte Relation verarbeitet werden muss. Bei einem Join können komplexere Situationen auftreten. Zuerst wird das Produkt der Kardinalität der Eingabe Parameter berechnet. Diese kann dann mit der 
Selektivität multipliziert werden. Das Produkt der Kardinalitäten repräsentiert in diesem Falle die Kardinalität des möglichen Kreuzproduktes, das mit Hilfe von Selektoren eingeschränkt wird.



\section{Search Space}

Der Suchraum (Search Space) bildet die Grundlage für die Suche nach einem optimalen Plan. Innerhalb des Suchraums findet die Auswahl des optimalen Plans für eine spätere Ausführung getroffen werden. Der Search Space wird im ersten Schritt erforscht. Pläne werden gefunden und später bewertet.

\begin{figure}[h]
  \centering
  \includegraphics[width=\textwidth]{02_Grundlagen/SearchSpace.png}
  \caption{Search Space}
  \label{SearchSpace}
\end{figure}


Als Search Space wird die Menge der logisch äquivalenten Pläne, die auf Grund einer Anfrage gebildet werden können, bezeichnet. Die Menge der äquivalenten Pläne kann so mächtig und so gross sein, dass nicht alle äquivalenten Pläne mit Hilfe von bekannten Techniken gefunden werden können. Der Search Space, der mit Hilfe dieser bekannten Mittel gefunden werden kann, wird als poteniteller Suchraum (Potential Search Space) bezeichnet. Die Pläne, die innerhalb dieses Search Spaces liegen, werden auf Grund ihrer Erreichbarkeit auch als accessable bezeichnet. Die Menge der Pläne, die nicht gefunden werden können heissen folglich non-accessable. Da selbst die Menge der Pläne des potential Search Spaces sehr gross sein kann, brechnen Anfragenoptimierer i.d.R. ihre Suche nach Planalternativen ab. Die Menge der tatsächlich gefundenen Alternativen wird als tatsächlicher Suchraum (actual Search Space) bezeichnet.

In Abbildung \ref{SearchSpace} ist der optimale Fall eines Search Spaces zu sehen. Der actual Search Space ist ein Subset des potential Search Spaces und dieser wiederum ein Subset des gesamten Search Spaces. In der Realität sind auch andere Formen möglich. Der Search Space kann beispielsweise durch eine fehlerhafte Implementierung Pläne dem actual Search Space zuordnen, die selbst nicht Teil des gesamten Search Spaces sind.

Um einen Search Space zu erforschen kommen Enumeratoren zum Einsatz. Diese werden im nächsten Abschnitt genauer behandelt.

\subsection{Enumeratoren}



Enumeratoren sind für die Erforschung eines Surchraums verantwortlich. Sie generieren basierend auf einer Anfrage unterschiedliche alternative Pläne. Es lässt sich im Allgemeinen zwischen zwei Klassen von Enumeratoren, die in tatsächlich genutzten Datenbanksystemen vorkommen, unterscheiden: Top-Down regelbasierte Optimierer wie Volcano und Cascades und Bottom-Up System-R ähnliche Optimierer mit dynamischer Programmierung. \cite{li2007enabling}

Das Konzept der dynamischen Programmierung wurde zuerst durch IBM's System R umgesetzt \cite{selinger1979access}. In diesem System wird die Reihenfolge der JOINs, die für eine Anfrage notwendig sind verändert, um die Anfrage zu optimieren. Der Search Space, der durch den Enumerator erforscht wird, besteht aus Join Trees. Je nach Enumerator werden nur bestimmte Join Trees behandelt.


\begin{figure}[h]
  \centering
  \includegraphics[scale=0.75]{02_Related_Work/TreeTypes.pdf}
  \caption{Tree Types}
  \label{TreeTypes}
\end{figure}

Je nach \ac{DBMS} kann sich ein Enumerator nur auf einen Teil des gesamten Search Spaces beschränken. So ist es möglich, dass eine Beschränkung auf Grund der Form eines Baumes gebildet wird. Grob lässt sich wie in Abb. \ref{TreeTypes} zu sehen ist, zwischen drei Arten von Bäumen unterscheiden, Left-Deep, Bushy und Right-Deep Trees. Je nach behandelter Join Art werden mehr oder weniger potenzielle Bäume aus dem Suchraum ausgeschlossen. Je nach Größe des potenziellen Suchraums kann sich die Suche effizienter oder weniger effizient gestalten. Für einige Systeme (System R) wird argumentiert, dass die Beschränkung auf beispielsweise Left-Deep Trees vorteilhaft wäre, da Aufwand für die Suche nach den Plänen gespart werden kann und trotzdem eine große Menge an alternativen Plänen erzeugt wird, die für sich genommen auch einen Plan nahe dem optimalen Plan enthält. Dieser Aussage wird jedoch durch \cite{ioannidis1991left} widersprochen. Er stellt fest, dass im Raum der Bush-Trees die besten Resultate der Optimierung gefunden werden können, da dort die meisten alternativen zur Verfügung stehen. 

\section{Kostenschätzung}
Um den optimalen Plan zu finden, müssen die Kosten für einen Plan bewertet werden. Deise Kostenabschätzung geschieht in einer Kostenschätzung \cite{bruno2011automated}. 

Die Kosten einer Datenbankabfrage sind von mehreren Faktoren abhängig. Um eine Grundlage für den Vergleich von verschiedenen Plänen zu haben, müssen die Kosten für jeden einzelnen Plan abgeschätzt werden. Die Kosten für die gesamte Abfrage setzten sich dabei aus den Kosten für die Enumeration der Abfrage, der Berechnung der Kosten und der Ausführung der eigentlichen Anfrage zusammen. Da die Kosten für Enumeration und Kostenabschätzung für jeden Plan gleich sind, kann das Augenmerk auf die Kostenschätzung für das physikalische Ausführend er Anfrage gelegt werden.

Bei der Ausführung einer Anfrage sind vor allem die CPU, die input / output und Arbeitsspeicherkosten entscheidend. Die Kosten können für jede dieser Ressourcen getrennt oder gemeinsam betrachtet werden. Die Kostenabschätzung muss insgesamt jedoch akkurat sein, da der \ac{QO} nur so gut sein kann, wie es der Kostenschätzer erlaubt. Die Berechnung der Kosten geschieht i.d.R. nach folgender Massgabe:

\begin{itemize}
\item Statistische informationen werden gesammelt
\item Für einen Operator im \ac{QEP} werden die Informationen aus Subbäumen berechnet und zusammengezalt
\end{itemize}

Die Kosten, die bei der Berechnung eines Baumsentstehen lassen sich mit Hilfe von Kostenfunktionen abhängig von der Datenmenge bestimmen. Daher sind die Kosten eines Plans stark von Kardinalitäten der Subpläne abhängig. Ebenso wie die Kardinalität spielt die Selektivität einzelner Operatoren eine grosse Rolle. Je selektiver ein Operator ist und je kleiner die Kardinalität der Subtrees ist desto schneller lässt sich eine Anfrage ausführen.

 






