\subsection{Implementierung von Pyro(J)}
\label{sec:pyroJ}
PyroJ ist der Optimierer, der von \cite{shanbhag2014optimizing} für die Prüfung der Vollständigkeit von Pellenkofts Regelmenge RS-B2 genutzt wurde. Dies wird im folgenden Kapitel genauer beschrieben. Ebenfalls wurde auf Basis dieses Optimierers die neue Regelmenge RS-Graph implementiert und getestet. 
PyroJ basiert auf dem von \cite{roy2001multi} in C++ Implementierten Optimierer Pyro und wurde automatisch von C++ nach Java übersetzt. Der Optimierer Pyro wurde nach dem Vorbild des Volcano Optimierers entwickelt. Volcano wurde als Vorbild gewählt, da es sich bei Volcano um einen hoch-respektierten, state-of-the-art, regelbasierten Optimierer handelt, der auch die Basis von kommerziellen Datenbanksytemen wie MS SQL Server ist. Gerade die Erweiterbarkeit in Hinblick auf das Datenmodell, Executionmodell und die Möglichkeit Transformationsregeln und Operatoren hinzuzufügen, sollten übernommen werden.


Einige Unterschiede zwischen Volcano und Pyro bestehen jedoch aus Sicht von \cite{roy2001multi}:

\subsubsection{Trennung zwischen logischem und physischen Planspace}

Pyro generiert zuerst einen logische Pläne, die dann in physiche Pläne umgesetzt werden und aus denen in einem dritten Schritt der optimale Plan ausgewählt werden kann. Diese Schritte werden nach einander unabhängig ausgeführt. In der Realität können diese drei Schritte überlappen. So ist es möglich, dass zuerst für einen logischen Plan alle physischen Pläne erzeugt werden und dann aus diesen Plänen nur der günstigste behalten wird. Daraufhin kann dann der nächste logische Plan erzeugt werden und für ihn der günstigste physische Plan gesucht werden. Nur wenn diese günstigste Plan billiger ist als der bisher gefundene Plan wird der phyische Plan auch weiterhin gespeichert. Dieser Ansatz kann Resourcen-schonender sein, als der in Pyro verwendete Ansatz, bei dem immer alle Daten vorgehalten werden. 

\subsubsection{Vereinigung von Äuqivalenten-Subausdrücken}

Bei Volcano ist es möglich gewesen, dass mehrere Äquivalenzklassen den selben Knoten repräsentieren. Beispielsweise kommt in der Anfrage $(A \Join B \Join C) \cup (B \Join C \Join D)$ $B$ und $C$ zweimal vor. Obwohl dies der Fall ist, werden für $B$ und für $C$ je zwei Äquivalenzklassen erzeugt. Nachdem diese beiden Relationen als unabhängig betrachtet werden, fällt auch nicht auf, dass der Ausdruck $B \Join C$ mehrfach vorkommt und somit auch in einer Äquivalenzklasse behandelt werden kann. Später wurde dieses Problem bei Volcano erkannt und mit Hilfe einer Memofunktion gelöst.

Auch Pyro(J) implementiert eine solche Memofunktion, die die Wiederverwendung von bekannten Äuqivalenzklassen erlaubt.


\subsubsection{Generierung mit Description-Files}

Ein fundamentaler Unterschied zwischen Volcano und Pyro auf den von \cite{roy2001multi} nicht hingewiesen wird, ist die Erstellung des Optimierers. Bei Pyro handelt es sich um einen fertigen Optimierer, der nicht mehr generiert werden muss. Es sind keine Description-Files vorhanden. Eine einfache Konfiguration sind nicht mehr möglich. Eine Generierung eines Optimierers findet überhaupt nicht statt.


\subsubsection{Ausführung der Experimente}

Auf PyroJ wurden von \cite{shanbhag2014optimizing} Experimente zur Überprüfung der Pellenkoft Regelmengen durchgeführt. Wie im folgenden Kapitel beschrieben, wurden Regelmengen auf ihre Vollständigkeit und Perfromance getestet. Ebenfalls wurde eine neue Regel implementiert. 
