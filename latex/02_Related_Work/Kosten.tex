\subsection{Kostenschätzung}

Um den optimalen Plan zu finden, müssen die Kosten für einen Plan bewertet werden. Diese Kostenabschätzung geschieht in einer Cost-Estimation \cite{bruno2011automated}. 

Die Kosten einer Datenbankabfrage sind von mehreren Faktoren abhängig. Um eine Grundlage für den Vergleich von verschiedenen Plänen zu haben, müssen die Kosten für jeden einzelnen Plan abgeschätzt werden. Die Kosten für die gesamte Abfrage setzten sich dabei aus den Kosten für die Enumeration der Anfrage, der Berechnung der Kosten und der Ausführung der eigentlichen Anfrage zusammen. Da die Kosten für Enumeration und Cost-Estimation für jeden Plan gleich sind und die Kosten für die Berechnung verglichen zu den Kosten des Auslesens der Daten wenig signifikant sind \cite{selinger1979access}, kann das Augenmerk auf die Abschätzung der Kosten der Ausführung gelegt werden.

Bei der Ausführung einer Anfrage sind vor allem die CPU, die Eingabe / Ausgabe und Arbeitsspeicherkosten entscheidend. Die Kosten können für jede dieser Ressourcen getrennt oder gemeinsam betrachtet werden. Die Kostenabschätzung muss insgesamt jedoch akkurat sein, da der Anfrageoptimierer nur so gut sein kann, wie es der Kostenschätzer erlaubt. Die Berechnung der Kosten geschieht i.d.R. nach folgender Maßgabe:

\begin{enumerate}
\item Statistische Informationen werden gesammelt
\item Für einen Operator im \ac{QEP} werden die Informationen aus Subbäumen berechnet und zusammengezählt
\end{enumerate}

Bei der Sammlung statistischer Informationen können neben der Kardinalität von Relationen, der Selektivität von Operatoren und andere Werte vorkommen. Falls noch keine Werte über ein bestimmtes Datum gesammelt wurde, werden i.d.R. Standardwerte angenommen. Am Ende ist der Plan optimal, der die niedrigsten Kosten aufweist. Einige Datenbanksysteme setzten bei der Berechnung der Kosten auf das Optimalitätsprinzip von Bellman \cite{Bellman:1957}, das aussagt, dass eine optimale Lösung immer aus optimalen Teillösungen besteht. Dieses Prinzip wird in diesem Kapitel im Abschnitt System R genauer besprochen und von Kapitel 5 Implementierung im Bereich der Kostenschätzung aufgegriffen.