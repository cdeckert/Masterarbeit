\chapter{Zusammenfassung}




\subsection{Ausblick}



Erweiterungen dieser Arbeit sind in mehreren Dimensionen möglich.Begonnen bei den hier implementierten Prototypen und der Tests, der zugrundeliegenden Arbeit.                                                                                                                                                   Wie bereits in Kapitel 5 kritisch bemerkt, kann Java als Plattform von Tests  bzgl. Performance leicht in Frage gestellt werden. Da mit Pyro eine Alternative zur Verfügung steht, können alle Tests nochmals mit diesem System nachvollzogen und so unabhängig geprüft werden. Teil dieser neuerlichen Prüfung kann hier auch eine Prüfung auf Vollständigkeit mit Hilfe aller Pläne sein, die wie in dieser Arbeit einzeln betrachtet und verglichen werden können. Ebenfalls ist eine Messung mit Kostenberechnung bei allen Algorithmen zu empfehlen, um ein vollständiges Bild zu zeichnen. Eine neue Implementierung von Graph Rule in Pyro kann die Ergebnisse erneut bestätigen.

Wie bereits in Kapitel 5 angesprochen, eröffnet die Kombination von bekannten Algorithmen und bekannten Optimierern neue Möglichkeiten zur Forschung. Weitere Algorithmen können implementiert werden und in neuem Kontext getestet werden. Eigenschaften verschiedener Systeme können ausgeglichen und völlig neu zusammengestellt werden. Andere Algorithmen wie \texttt{TDMINCUTLAZY} bieten ähnliches Potenzial zur Implementierung.

Einige technische Erweiterungen des Prototypen sind ebenfalls anzudenken: Zunächst  können durch die Implementierung nach den in Kapitel 4 vorgestellten SOLID-Prinzipien schnell neue Features hinzugefügt werden. Insbesondere kann eine Erweiterung von Adaptern sinnvoll sein, um so den Prototypen an andere Systeme anzuschließen und so noch mehr Funktionalität und Eigenschaften hinzuzufügen. Des weiteren können völlig neue Regeln und Regelmengen kombiniert und erstellt werden. Auch in diesem Bereich gibt es Potenziale, die es auszuschöpfen gilt. Die Implementierung verschiedener Kostenschätzer, die eine komplexere Kostenschätzung ermöglichen, würde den Prototypen um ein wichtiges Feature erweitern. 

