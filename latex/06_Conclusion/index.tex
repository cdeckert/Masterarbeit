\chapter{Ausblick und Fazit}


In dieser Arbeit wurde sowohl die Frage nach der Vollständigkeit der von Pellenkoft vorgestellten Regelmengen geprüft, als auch die neue, von \cite{shanbhag2014optimizing} implementierten Regelmenge RS-Graph mit ihrer Regel GraphRule untersucht und auf ihre Performance hin geprüft.

Zu diesem Zweck wurden in Kapitel 2 die Grundlagen gelegt und unterschiedliche Datenbankoptimierer und Datenbanksysteme zusammen mit Grundbegriffen angesprochen. In Kapitel 3 wurden die Regelmengen vorgestellt und die bisherige Forschung bzgl. der Performance und Vollständigkeit der Regelmengen dargelegt. Die für die Evaluation (Kapitel 5) notwendige Implementierung eines Prototypen ist in Kapitel 4 dokumentiert worden. Die Ergebnisse der Evaluation wurden in Kapitel 5 dargestellt.

Die Ergebnisse auf die Frage der Vollständigkeit der Pellenkoft Regelmengen zeigt, dass die Regelmenge RS-B2 zwar unvollständig ist, jedoch nicht im selben Maß wie von \cite{shanbhag2014optimizing} angenommen. Es wurde eine mögliche Begründung für die Messfehler von \cite{shanbhag2014optimizing} vorgeschlagen: Ein ungeeigneter Expansion Algorithmus. Abgesehen von diesem Algorithmus konnte in einer Menge von Tests gezeigt werden, dass RS-B2 bei einer Reihe von Formen korrekte Ergebnisse liefert und dies sogar ohne die Erzeugung von Duplikaten.

In einem zweiten Schritt konnte die Performance von RS-Graph geprüft werden. Wie zu erwarten, war RS-Graph in allen Fällen schneller als andere Regelmengen. Selbst die von \cite{shanbhag2014optimizing} betrachtete Regelmenge könnte RS-Graph übertreffen. Es wurde zudem festgestellt, dass die Dauer der Enumeration auch abhängig vom jeweiligen transformationsbasierten Enumerationsalgorithmus ist und ein solcher Algorithmus zu erheblichen unterschieden in der Laufzeit führen kann.

\section{Ausblick}

Die Arbeit eröffnet eine Vielzahl an Erweiterungsmöglichkeiten. Auf der einen Seite wurden bisher nur die Regelmengen von Pellenkoft und die neue Regelmenge RS-Graph untersucht. Andere Regelmengen wurden nicht untersucht. Neue Regelmengen insbesondere Regelmengen, die effizienter als die bisher vorhandenen Regelmengen sind, können erforscht und mit dem bestehenden Framework an Optimizer getestet werden.

In Kapitel 5 wurde erneut gezeigt, dass RS-B2 nicht vollständig ist. Es wurden Belege für die Unvollständigkeit gesammelt. Ein genauer Beweis der Unvollständigkeit und die Eingrenzung der Fälle steht aus.

Neben der Erweiterung auf andere transformationsbasierte Regeln, bietet es sich auch an andere Enumerationsalgorithmen in bestehende regelbasierte Transformationsregeln einzubauen. Beispielsweise liegt es nahe auch DPccp in das bestehende System einzubinden, um so die Geschwindigkeit der Enuemration weiter zu erhöhen.

Ebenfalls wurde in Kapitel 5 angesprochen, dass der verwendete Enumerator ExpandDAG zwar vergleichsweise schnell, jedoch nicht sehr akkurat bei der Enumeration von Plänen mit der Regelmenge RS-B2 ist. Der selbst implementierte Enumerator ist zwar viel akkurater, aber dafür externem ineffizient und repetitiv. Die Entwicklung von transformationsbasierten Enumerationsalgorithmen, die sich adaptiv auf die Bedürfnisse der Regelmengen anpassen und so den Ressourcenbedarf reduzieren, ist ein anderes Feld, dass es zu untersuchen gilt.  



\section{Fazit}

Abschließend lässt sich zusammenfassen, dass Regelmengen wie die von Pellenkoft zwar leicht eingängig, jedoch wenig effizient sind (im Vergleich zu MinCutConservative). Auch ist es möglich, dass Regelmengen nicht den gesamten Suchraum erforschen und so nicht das optimale Ergebnis finden. 

Regelbasierte Optimierer sind erweiterbar. Es ist möglich andere Enumeratoren als Regel zu verpacken und in einen regelbasierten Enumerator einzubauen. Ob es sich bei einem solchen Vorgehen um eine optimale Lösung handelt, liegt im Auge des Betrachters.





