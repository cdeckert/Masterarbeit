\section{Oracle}
\subsection{Oracle Architektur}


\begin{figure}[h]
  \centering
  \includegraphics[width=\textwidth]{03_Related_Work/OracleArchitecture.png}
  \caption{Oracle Architecture \cite{Oracle2004Basics}}
  \label{OracleArchitecture}
\end{figure}



Die Oracle Architektur zur Verarbeitung von Anfragen \cite{Oracle2004Basics} (vgl. Abb. \ref{OracleArchitecture})  beginnt wie die meisten Systeme mit einer SQL Anfrage, die von einem Parser in eine Interne Repräsentation gebracht wird. Der Parser übernimmt dabei zwei Funktionen. Auf der einen Seite die Syntaktische Analyse. Es wird geprüft, ob die SQL Anfrage die korrekte Syntax besitzt. Auf der anderen Seite eine Semantische Analyse. Diese prüft beispielsweise ob Datenbank Objekte und Objekt-Attribute  korrekt referenziert werden. Nachdem diese beiden Schritte ausgeführt wurden unterscheidet der Optimierer ob ein \ac{RBO} oder ein \ac{CBO} zum Einsatz kommt. Mit Version 11g der Oracle Datenbank wird der \ac{RBO} nicht mehr unterstützt und ausschließlich auf den \ac{CBO} gesetzt \cite{dba_oracle2015}.

Der \ac{QO} führt bei der Verarbeitung die folgenden drei Schritte aus:

\begin{itemize}
\item Eine Menge potentieller Pläne wird basierend der SQL Anfrage selbst und Hinweisen, die durch den Nutzer eingegeben werden, generiert.
\item Der Optimierer schätzt die Kosten für jeden Plan basierend auf Kosteninformationen über die Anfrage und Storage Charakteristiken der Tabellen, Indexe und Partitionen, die durch ein Statement zugegriffen werden können.

Die Kostenschätzung für den Zugriff auf entsprechende Datensätze und die Reihenfolge der Joins wird nach ihrem Verbrauch von Ressourcen wie I/O, CPU und Memory geschätzt.

Pläne mit höheren Kosten benötigen mehr Zeit zur Ausführung als Pläne mit niedrigeren Kosten. Falls Pläne parallel ausgeführt werden können, ist die Ressourcen Nutzung nicht direkt abhängig von der verwendeten Zeit.

\item Der Optimierer vergleicht die Kosten der Pläne und wählt den kostengünstigsten Plan aus.
\end{itemize}

Das Ergebnis des Optimizers ist ein Plan, der zum Ausführen der Anfrage geeignet ist und als kostenoptimal eingeschätzt wird.


\subsection{Kostenbasierte Transformation in Oracle}

Traditionelle Relationale Datenbank Systeme führen, wie bereits dargestellt, die Transformation von Anfragen in zwei Phasen durch: Logische und physische Phase. In der logischen Phase wird die gegebene Anfrage zuerst durch einen Rewriter angewendet, hier kommen Heueristiken oder Regeln zum Einsatz. Der traditionelle pyhsische Optimizer arbeitet mit einem einzelnen Query Block aus Restriktionen über Tabellen, Projektionen und Joins. Die Physische Optimizerungsphase befasst sich mit Access Methoden, Join Orders und join Methoden die genutzt werden, um effiziente Pläne zu erzeugen.

Query Transformation werden entweder als Rewriter Systeme oder als Extention des Plan generators mit dem pyhsichen Optimierer implementiert. Der erste Ansatz skaliert nicht in komplexen kommerziellen Systemen, der zweite Ansatz ist nur einfach auf ein paar wenige Transformationen anzuwenden. 
Bisher wurden zuerst Rewriter Rules angewendet und dann die Nutzung von Cost Based Trasnformations angewendet. \cite{ahmed2006cost} argumentiert, dass einige Heueristiken nicht immer ein optimales Ergebnis erzeugen und daher mit Hilfe von Cost Based Methoden geprüft werden sollten, bevor sie auf eine Anfrage angewendet werden.


Oracle hat die kostenbasierte Transformation, die logische Transformation und die physische Optimierung kombiniert, um den optimalen Execution Plan zu finden. Der Logische Teil des Systems teilt sich in Heueristiken und kostenbasierte Transformationen. Die kostenbasierten Trasnformatioenen funktionieren wie folgt:

\begin{itemize}
\item Transformationsalorithmen konvertieren gesamte under teil Anfrage Bäume in sequentiell gleiche Form
\item Stae Spaces für verschiedene Transformationen
\item State Space Search Algorithmen
\item Möglichkeit zur Deep Copy von Anfrage Böcken und Ihrer Consitutues
\item Cost estimation techique (physical optimizer)
Transformation derective und cost annotations
\end{itemize}

Unterschiedliche Transformationsregeln werden auf unterschiedliche Teile der Anfrage angewendet. So kann beispielsweise die Entschachtelung nur auf verschachtelte Elemente der Anfrage angewendet werden. 

Die Bäume werden Bottom up transformiert während der Optimierung. Verschiedene Alternativen für eine oder mehrere Transformationen für Elemente in einem Query Tree generieren unterschiedliche States innerhalb des Ste Space of Transformation. Eine Deep kopy wird gemacht bevor ein bestimmter status erreicht wird und dessen Kosten durch den Aufruf des physischen Optimierers geschätzt werden. Die Evaluation von jedem States. Der Transformationsstage, der die besten Kosten bietet wird zurück auf den Ursprungstree angewendet.

Die Oracle Transformationen werden immer hintereinander angewendet. Eine Transformatiosnregel wird immer auf den ganzen Baum angewendet, erst danach folgt die nächste Regel. Die Reihenfolge ist diese: Common sub expression factorization, JPS view merging, join elimination, subquery unnesting, group by distinct view merging, group pruning, predicate move around, set operator factorization, disjunction into union-all exansion, star transfomration and join predicate pushdown. Von dieser Reihenfolge kann jedoch in bestimmten Fällen abgewichen werden. 



\subsubsection{State Space Search Techniques}
Bei der Suche nach einem Plan mit Hilfe von kostenbasierender Transformation stellt sich die grundsätzliche Frage nach einem Trade-off zwischen optimalen Kosten und Execution Kosten. Die Suche nach einer Transformation lohnt sich nur dann, wenn die Kosten für die Suche nach einer besseren Transformation gemeinsam mit der Ausführung der eigentlichen Anfrage geringer ist als die bisher gefundenen Pläne.

Die Frage nach diesem Problem stellt sich insbesondere, wenn eine große Menge an möglichen alternativen Plänen bestehen.. Existieren viele einzelne Objekte einer Anfrage, seien es Query Blocks, Tables join edges predikate, gibt es auch die Möglichkeit, dass viele Regeln auf die Objekte angewendet und somit viele alternative Pläne erzeugt werden können. Wenn $N$ Objekte vorhanden sind und auf diese $N$ Objekte eine Transformation $T$ angewendet wird, dann werden nur durch diese eine transformation $2^N$ verschiedene Pläne generiert. Dies Menge wächst weiter, wenn mehrere Transformationen auf die bestehenden Objekte angewendet werden können. 


Um dieses kombinatorische Problem der vielen oin permutationen zu lösen wurden mehrere randomizierte Algorithmen vorgeschlagen. Tabu Search, Genetic Search, Iterative Improvement...

chst mit der Anzahl der Trasnfromationsobjekte. Wenn die Anzahl der Transformationsobjekte klein ist, dann ist eine enumerative Transformationstechnik mit Hilfe Exhaustive search des States vielleicht machbar. Da aber die Anzahl der möglichen Optimierungen mit der Zeit steigt, müssen andere Techniken angewendet werden:

\begin{itemize}
\item 
\end{itemize}

\subsubsection{Subquery Entschachtelung}

\subsubsection{Join Elimination}

\subsubsection{Filter Predicate Move Around}

\subsubsection{Group Pruning}
