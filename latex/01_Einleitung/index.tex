\chapter{Einleitung}

\section{Motivation}
Anfragenoptimierung ist eine der wichtigsten Bestandteile bei der Ausführung von Datenbankabfragen. Insbesondere im Kontext von komplexen und automatisch generierten SQL-Statements verspricht die Optimierung hohe Geschwindigkeitsverbesserungen. Der Optimierer ist für das Finden der richtigen Zugriffsstrategie auf Datensätze verantwortlich. Er entscheidet u.a. welche Art von Joins verwendet werden, in welcher Reihenfolge diese Joins angewendet werden oder ob Indexe anstatt von Full table scans zum Einsatz kommen. Nur durch die erreichte Performance heute gängiger \ac{DBMS} sind Anwendungen und Informationssysteme wie \ac{ERP}, \ac{CRM}, \ac{CMS}, \ac{BDWH} möglich geworden. \ac{DBMS} sind ein integraler Bestandteil unserer heutigen Wissens- und Informationsgesellschaft. Der schnelle und effiziente Zugriff auf Informationen und damit die Nutzung von \ac{DBMS} ist aus unserer Welt nicht mehr wegzudenken. Anfragenoptimierung bleibt daher ein wichtiger Teil.




\section{Ziel der Arbeit}

\section{Inhaltlicher Aufbau}
